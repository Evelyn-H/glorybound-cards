%%% Commands to be used in card definitions

% \newcommand{\block}{{\par\centering\textbf{Block} an attack.\par}}

% \newcommand{\attack}[1]{{\par\centering\textbf{Attack} #1.\par}}

\pgfkeys{/fancyrenderer/.cd,% set the initial path
    text/.initial={},
    icon/.initial={},
    icontext/.code={\pgfkeyssetvalue{/fancyrenderer/icontext}{#1}},
    fillcolor/.initial={white},
    textcolor/.initial={black},
    scale/.initial={0.8},
    % scale/.initial={0.8},
}

% helper lengths
\newlength\cornersize
\newlength\iconwidth
\newlength\tempfontsize
\newlength\outlinewidth

% https://tex.stackexchange.com/a/6424
\makeatletter
\newcommand*{\DivideLengths}[2]{%
  \strip@pt\dimexpr\number\numexpr\number\dimexpr#1\relax*65536/\number\dimexpr#2\relax\relax sp\relax
}
\makeatother

\NewDocumentCommand{\fancyrenderer}{ +O{} }{%
    % some magic for autoscaling
    \pgfkeyssetvalue{/fancyrenderer/scale}{%
        \directlua{%
            ratio = \DivideLengths{\tcbfitdim}{\pgfkeysvalueof{/textbox/fontsize}}
            % tex.print(ratio)
            if ratio > 0.9 then
                tex.print("1.0")
            % elseif ratio < 0.7 then
            %     tex.print("0.6")
            else
                tex.print("0.8")
            end
        }
    }
    % \pgfkeyssetvalue{/fancyrenderer/scale}{\DivideLengths{\tcbfitdim}{\pgfkeysvalueof{/textbox/fontsize}}}
    % horizontal centering (relative to the parent parbox)
    \centering    
    % to make sure no arguments "leak" between different calls
    \begingroup
    % handle key-value arguments
    \pgfkeys{/fancyrenderer/.cd, #1}
    \begin{tikzpicture}[x=1mm, y=1mm]
        \node[
            anchor=center,
            rectangle,
            % minimum width=0.8\linewidth, 
            minimum width=36mm, 
            minimum height=6mm*\pgfkeysvalueof{/fancyrenderer/scale},
            % fill=gray, 
        ] (bb) {};


        \setlength\cornersize{1.5mm*\real{\pgfkeysvalueof{/fancyrenderer/scale}}}
        \path[
            draw=black, line width=0.5mm, 
            fill=\pgfkeysvalueof{/fancyrenderer/fillcolor}
        ] 
            ([xshift=-\cornersize, yshift=-0.25mm] bb.north east) coordinate (start)
            --
            ([xshift=\cornersize, yshift=-0.25mm] bb.north west)
            --
            ([xshift=0.25mm, yshift=-\cornersize] bb.north west)
            -- 
            ([xshift=0.25mm, yshift=\cornersize] bb.south west)
            --
            ([xshift=\cornersize, yshift=0.25mm] bb.south west)
            -- 
            ([xshift=-\cornersize, yshift=0.25mm] bb.south east)
            -- 
            (start);

        \setlength\tempfontsize{10pt*\real{\pgfkeysvalueof{/fancyrenderer/scale}}}

        \node[
            anchor=west,
            % anchor=mid west,
            % minimum width=0.8\linewidth-8mm,
            minimum width=36mm-8mm,
        ] at (bb.west) {
            \fontsize{\tempfontsize}{\tempfontsize}\selectfont
            \textcolor{\pgfkeysvalueof{/fancyrenderer/textcolor}}{\pgfkeysvalueof{/fancyrenderer/text}}
        };

        \setlength\iconwidth{10mm*\real{\pgfkeysvalueof{/fancyrenderer/scale}}}

        % attack icon
        \node[
            anchor=east, 
            inner ysep=0pt,
            inner xsep=0pt,
            outer ysep=0pt,
            outer xsep=0pt,
        ] (icon) at (bb.east) {
            \includegraphics[width=\iconwidth]{\geticon{\pgfkeysvalueof{/fancyrenderer/icon}}{\pgfkeysvalueof{/card/color-left}}{\pgfkeysvalueof{/card/color-right}}}
        };

        % and the white-ish semi-transparent overlay to make the colors a bit more wahsed out
        \node[
            anchor=east, 
            inner ysep=0pt,
            inner xsep=0pt,
            outer ysep=0pt,
            outer xsep=0pt,
            opacity = 0.5,
        ] at (bb.east) {
            \includegraphics[width=\iconwidth]{\geticon{\pgfkeysvalueof{/fancyrenderer/icon}}{FFFFFF}{FFFFFF}}
        };

        \setlength\tempfontsize{24pt*\real{\pgfkeysvalueof{/fancyrenderer/scale}}}
        \setlength\outlinewidth{0.75mm*\real{\pgfkeysvalueof{/fancyrenderer/scale}}}

        \pgfkeysifdefined{/fancyrenderer/icontext}{
            % https://tex.stackexchange.com/questions/400296/outline-text-characters
            \node[
                anchor=center,
                minimum width=\iconwidth, 
                minimum height=\iconwidth,
            ] at (icon.center) {%
                \fontsize{\tempfontsize}{\tempfontsize}\selectfont
                \bfseries
                \textpdfrender{
                    TextRenderingMode=FillStrokeClip,
                    LineWidth=\outlinewidth,
                    FillColor=white,
                    StrokeColor=black, 
                    MiterLimit=1,
                }{\pgfkeysvalueof{/fancyrenderer/icontext}}
            };
            \node[
                anchor=center,
                minimum width=\iconwidth, 
                minimum height=\iconwidth,
            ] at (icon.center) {%
                \fontsize{\tempfontsize}{\tempfontsize}\selectfont
                \bfseries
                \textcolor{white}{\pgfkeysvalueof{/fancyrenderer/icontext}}
            };
        }{}
        
    \end{tikzpicture}%
    \endgroup
}

\newcommand{\block}{%
    \fancyrenderer[%
        text={\textbf{Block} an attack.},
        icon=block,
        fillcolor=black,
        textcolor=white,
    ]
}
\newcommand{\attack}[1]{%
    \fancyrenderer[%
        text={\textbf{Attack}},
        icon=attack,
        icontext={#1},
    ]
}

\renewcommand\tabularxcolumn[1]{m{#1}}% for vertical centering text in X column
\renewcommand{\arraystretch}{1.5}
\NewDocumentCommand{\sequence}{+o +o +o}{%
    \begin{tabularx}{\textwidth}{ m{3mm}>{\raggedright\arraybackslash}X }
        [1] & {#1} \\
        \IfValueT{#2}
        {%
            [2] & {#2} \\
            \IfValueT{#3}
            {%
                [3] & {#3} \\
            }
        }
    \end{tabularx}
}

%%%


%%% Misc rendering helper functions

\newcommand{\rendermanasymbol}[2]{
    \directlua{
        function get_mana_filename(s)
            symbols = {}
            symbols["A"] = "any"
            symbols["F"] = "focus"
            symbols["S"] = "strength"
            symbols["W"] = "will"
            % symbols["S"] = "spirit"
            return symbols[s] or s
        end
    }
    \tikz[overlay, remember picture] 
    \node[
        anchor=north west, 
        inner ysep=0pt,
        inner xsep=0pt,
        outer ysep=0pt,
        outer xsep=0pt,
        xshift=5.5mm,
        yshift=-\directlua{tex.print(5.5 + 6 * (#1 - 1))}mm,
    ] at (current page.north west) {
        % \includesvg[width = 63mm]{frames/frame-berserker.svg}
        \includegraphics[width=5mm]{icons/mana/\directlua{tex.print(get_mana_filename(\luastringN{#2}))}.png}
    };
}

\begin{luacode*}
    function render_mana(s)
        for i = 1, #s do
            tex.print("\\rendermanasymbol{"..i.."}{"..s:sub(i,i).."}")
        end
    end
\end{luacode*}
\newcommand{\rendermana}[1]{\directlua{render_mana(\luastring{#1})}}


% building the attack and block icons
% generate gradient png
% magick -size 474x473 -define gradient:direction=Northeast gradient:"#00DB1D"-"#BF00FF" gradient.png

% two commands for the overlaying
% convert gradient.png attack.png -composite resulta.png
% convert resulta.png attack-mask-2.png  -compose copy_opacity -composite result.png

% and this just combines both of them into one
% convert gradient.png attack.png -composite attack-mask.png -compose copy_opacity -composite result.png

% and then this combines all three into one:
% convert -size 474x473 -define gradient:direction=Northeast gradient:"#00DB1D"-"#BF00FF" attack.png -composite attack-mask.png -compose copy_opacity -composite result.png


\begin{luacode*}
    function generate_icon(name, col1, col2)
        io.popen("cd icons/ && convert -size 473x473 -define gradient:direction=Northeast gradient:\"#"..col1.."\"-\"#"..col2.."\" "..name..".png -composite "..name.."-mask.png -compose copy_opacity -composite "..name.."-"..col1.."-"..col2..".png")
    end
    function get_icon(name, col1, col2)
        filename = "icons/"..name.."-"..col1.."-"..col2..".png"
        -- check if it already exists
        local f = io.open(filename, "r")
        if f~=nil then 
            io.close(f) 
            return filename
        else 
            generate_icon(name, col1, col2)
            return filename  
        end
    end
\end{luacode*}
\newcommand{\geticon}[3]{\directlua{tex.print(get_icon(\luastring{#1}, \luastring{#2}, \luastring{#3}))}}
