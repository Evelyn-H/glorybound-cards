\colorlet{dark}{black}
\colorlet{light}{white}
% \colorlet{dark}{red}
% \colorlet{light}{green}
% \colorlet{dark}{white}
% \colorlet{light}{black!90}


%%% Commands to be used in card definitions

\colorlet{ugradegray}{dark!25!light!75}
% \newcommand{\upgrade}[1]{\sethlcolor{ugradegray}\hl{{#1}}}
% \newcommand{\upgrade}[1]{\colorbox{ugradegray}{ugradegray}{#1}}

% \newcommand{\upgrade}[1]{%
% % #1%
%     \setlength{\fboxsep}{0pt}%
%     \colorbox{ugradegray}{%
%         % \vphantom{Blg}%
%         \strut%
%         #1%
%         % \makebox{something}%
%         % \makebox{
%         %     \strut#1
%         % }
%     }%
% }

\makeatletter
\newcommand{\upgrade}[2]{%
% #1%
    \setlength{\fboxsep}{0pt}%
    \colorbox{ugradegray}{%
        % \vphantom{Blg}%
        \strut%
        \,#1\,%
        % \makebox{something}%
        % \makebox{
        %     \strut#1
        % }
    }%
    \ifx\upgradereplacement\empty%
    \else%
        \g@addto@macro{\upgradereplacement}{;\ }%
    \fi%
    \g@addto@macro{\upgradereplacement}{\{#2\}}%
}
\makeatother

% \newcommand{\upgrade}{\bgroup\markoverwith{\textcolor{ugradegray}{\rule[-.5ex]{.5pt}{2.5ex}}}\ULon}

% \newcommand{\block}{{\par\centering\textbf{Block} an attack.\par}}

% \newcommand{\attack}[1]{{\par\centering\textbf{Attack} #1.\par}}

\pgfkeys{/fancyrenderer/.cd,% set the initial path
    text/.initial={},
    icon/.initial={},
    icontext/.code={\pgfkeyssetvalue{/fancyrenderer/icontext}{#1}},
    fillcolor/.initial={light},
    textcolor/.initial={dark},
    scale/.initial={1.0},
    % scale/.initial={0.8},
}

% helper lengths
\newlength\boxwidth
\newlength\cornersize
\newlength\iconwidth
\newlength\tempfontsize
\newlength\outlinewidth

% https://tex.stackexchange.com/a/6424
\makeatletter
\newcommand*{\DivideLengths}[2]{%
  \strip@pt\dimexpr\number\numexpr\number\dimexpr#1\relax*65536/\number\dimexpr#2\relax\relax sp\relax
}
\makeatother

\NewDocumentCommand{\fancyrenderer}{ +O{} }{%
    % to make sure no arguments "leak" between different calls
    \begingroup%
    % some magic for autoscaling
    % \pgfkeyssetvalue{/fancyrenderer/scale}{\DivideLengths{\tcbfitdim}{\pgfkeysvalueof{/textbox/fontsize}}}%
    \pgfkeyssetvalue{/fancyrenderer/scale}{%
        \directlua{%
            ratio = \DivideLengths{\tcbfitdim}{\pgfkeysvalueof{/textbox/fontsize}}
            % tex.print(ratio)
            if ratio > 0.9 then
                tex.print("1.0")
            % elseif ratio < 0.7 then
            %     tex.print("0.6")
            else
                tex.print("0.8")
            end
        }
    }%
    \setlength\boxwidth{36mm}%
    % horizontal centering (relative to the parent parbox)
    \ifdefined\insidesequence
        \pgfkeyssetvalue{/fancyrenderer/scale}{0.8}%
        \setlength\boxwidth{32mm}%
    \else
        \centering%
    \fi
    % handle key-value arguments
    \pgfkeys{/fancyrenderer/.cd, #1}%
    \begin{tikzpicture}[x=1mm, y=1mm]
        \node[
            anchor=center,
            rectangle,
            % minimum width=0.8\linewidth, 
            minimum width=\boxwidth, 
            minimum height=6mm*\pgfkeysvalueof{/fancyrenderer/scale},
            % fill=gray, 
        ] (bb) {};


        \setlength\cornersize{1.5mm*\real{\pgfkeysvalueof{/fancyrenderer/scale}}}
        \path[
            draw=dark, line width=0.5mm, 
            fill=\pgfkeysvalueof{/fancyrenderer/fillcolor}
        ] 
            ([xshift=-2mm, yshift=-0.25mm] bb.north east)
            --
            ([xshift=\cornersize, yshift=-0.25mm] bb.north west)
            --
            ([xshift=0.25mm, yshift=-\cornersize] bb.north west)
            -- 
            ([xshift=0.25mm, yshift=\cornersize] bb.south west)
            --
            ([xshift=\cornersize, yshift=0.25mm] bb.south west)
            -- 
            ([xshift=-2mm, yshift=0.25mm] bb.south east)
            -- 
            cycle;

        \setlength\tempfontsize{10pt*\real{\pgfkeysvalueof{/fancyrenderer/scale}}}

        \node[
            anchor=west,
            % anchor=mid west,
            % minimum width=0.8\linewidth-8mm,
            minimum width=\boxwidth-8mm,
        ] at (bb.west) {
            \fontsize{\tempfontsize}{\tempfontsize}\selectfont
            \textcolor{\pgfkeysvalueof{/fancyrenderer/textcolor}}{\pgfkeysvalueof{/fancyrenderer/text}}
        };

        \setlength\iconwidth{10mm*\real{\pgfkeysvalueof{/fancyrenderer/scale}}}

        % attack icon
        \node[
            anchor=east, 
            inner ysep=0pt,
            inner xsep=0pt,
            outer ysep=0pt,
            outer xsep=0pt,
        ] (icon) at (bb.east) {%
            \includegraphics[width=\iconwidth]{\geticon{\pgfkeysvalueof{/fancyrenderer/icon}}{\pgfkeysvalueof{/card/color-left}}{\pgfkeysvalueof{/card/color-right}}}%
        };

        % and the white-ish semi-transparent overlay to make the colors a bit more wahsed out
        \node[
            anchor=east, 
            inner ysep=0pt,
            inner xsep=0pt,
            outer ysep=0pt,
            outer xsep=0pt,
            opacity = 0.5,
        ] at (bb.east) {%
            \pgfkeysifdefined{night}{%
                \includegraphics[width=\iconwidth]{\geticon{\pgfkeysvalueof{/fancyrenderer/icon}}{000000}{000000}}%
            }{%
                \includegraphics[width=\iconwidth]{\geticon{\pgfkeysvalueof{/fancyrenderer/icon}}{FFFFFF}{FFFFFF}}%
            }%
        };

        \setlength\tempfontsize{24pt*\real{\pgfkeysvalueof{/fancyrenderer/scale}}}
        \setlength\outlinewidth{0.75mm*\real{\pgfkeysvalueof{/fancyrenderer/scale}}}

        \pgfkeysifdefined{/fancyrenderer/icontext}{
            % https://tex.stackexchange.com/questions/400296/outline-text-characters
            \node[
                anchor=center,
                % minimum width=\iconwidth, 
                % minimum height=\iconwidth,
                % fill=gray,
            ] at (icon.center) {%
                \fontsize{\tempfontsize}{\tempfontsize}\selectfont
                \bfseries
                \textpdfrender{
                    TextRenderingMode=FillStrokeClip,
                    LineWidth=\outlinewidth,
                    FillColor=light,
                    StrokeColor=dark, 
                    MiterLimit=1,
                }{\pgfkeysvalueof{/fancyrenderer/icontext}}%
            };
            \node[
                anchor=center,
                % minimum width=\iconwidth, 
                % minimum height=\iconwidth,
            ] at (icon.center) {%
                \fontsize{\tempfontsize}{\tempfontsize}\selectfont
                \bfseries
                \textcolor{light}{\pgfkeysvalueof{/fancyrenderer/icontext}}%
            };
        }{}

        % this just draws a little line over the left edge of the icon
        % to try and hide the transparancy issues there
            
        \ifthenelse{\equal{\directlua{
            icon = \luastring{\pgfkeysvalueof{/fancyrenderer/icon}}
            texio.write('asdf'..icon)
            if icon:find('^block') == nil then
                tex.sprint('true')
            else
                tex.sprint('false')
            end
        }}{false}}{
            \path[
                draw=dark, line width=0.5mm, 
            ] 
                ([xshift=-\iconwidth*0.9, yshift=-0.0mm] bb.north east)
                --    
                ([xshift=-\iconwidth*0.9, yshift=-0.1mm] bb.north east)
                --    
                ([xshift=-\iconwidth] bb.east)
                --
                ([xshift=-\iconwidth*0.9, yshift=0.1mm] bb.south east)
                --
                ([xshift=-\iconwidth*0.9, yshift=0.0mm] bb.south east);
        }{}

        
    \end{tikzpicture}%
    \par
    \endgroup%
}

\newcommand{\block}{%
    \fancyrenderer[%
        text={\textbf{Block} an attack.},
        icon=\pgfkeysifdefined{night}{block-night}{block},
        fillcolor=dark,
        textcolor=light,
    ]
}
\newcommand{\attack}[1]{%
    \fancyrenderer[%
        text={\textbf{Attack}},
        icon=\pgfkeysifdefined{night}{attack-night}{attack},
        icontext={#1},
    ]
}

\newlength{\turncountersize}
\setlength{\turncountersize}{4.8mm}

\colorlet{sequencecolor}{dark!70}

\newcommand{\sequencerowsep}{%
    \node[
        anchor=north east,
        inner sep=0pt,
        minimum width=\linewidth, 
        minimum height=2mm,
    ] (end) at (text.south east) {};

    % \path[
    %     draw=sequencecolor, line width=0.25mm, 
    % ]
    % ([xshift=9mm] end.west) -- ([xshift=-9mm] end.east);
}
\newcommand{\sequencerow}[2][]{%
    % rules text
    \node[
        anchor=north east,
        % anchor=mid west,
        text width=\linewidth-6mm-3mm,
        minimum height=\turncountersize,
        inner sep=0pt,
        align=flush left,
        % fill=sequencecolor,
    ] (text) at (end.south east) {%
        \def\insidesequence{}%
        \fontsize{\tcbfitdim}{\tcbfitdim}\selectfont%
        % \color{light}
        #2
    };

    % turn counter 
    \node[
        fit={(text.north west) (text.south west)},
        anchor=east,
        rectangle,
        inner sep=0pt,
        minimum width=\turncountersize, 
        minimum height=\turncountersize,
        xshift=-3mm,
        % fill=sequencecolor,
    ] (turn) at (text.west) {};

    \path[
        draw=sequencecolor, line width=0.5mm, 
        fill=sequencecolor,
    ] 
        ([xshift=-1.2mm, yshift=-0.25mm] turn.north east)
        --
        ([xshift=1.2mm, yshift=-0.25mm] turn.north west)
        --
        ([xshift=0.25mm, yshift=-1.2mm] turn.north west)
        -- 
        ([xshift=0.25mm, yshift=1.2mm] turn.south west)
        --
        ([xshift=1.2mm, yshift=0.25mm] turn.south west)
        -- 
        ([xshift=-1.2mm, yshift=0.25mm] turn.south east)
        -- 
        ([xshift=-0.25mm, yshift=1.2mm] turn.south east)
        -- 
        ([xshift=-0.25mm, yshift=-1.2mm] turn.north east)
        --
        cycle;

    \node[
        anchor=center,
    ] at (turn.center) {%
        \fontsize{10pt}{10pt}\selectfont
        \textcolor{light}{\textbf{#1}}%
    };
}

\NewDocumentCommand{\sequence}{+o +o +o +o}{%
    \begin{tikzpicture}[x=1mm, y=1mm]
        \node[
            inner sep=0pt,
            minimum width=0mm, 
            minimum height=0mm,
            fill=sequencecolor, 
        ] (end) {};

        \sequencerow[I]{#1}
        % this node just saves the location of the first turn counter
        % so we can use it for drawing the timeline
        \node[
            anchor=north west,
            inner sep=0pt,
            minimum width=\turncountersize, 
            minimum height=0mm,
            fill=sequencecolor, 
            ] (topturn) at (turn.north west){};
        % add as many rows as we need
        \IfValueT{#2}
        {%
            \sequencerowsep
            \sequencerow[II]{#2}
            \IfValueT{#3}
            {%
            \sequencerowsep
                \sequencerow[III]{#3}
                \IfValueT{#4}
                {%
                    \sequencerowsep
                    \sequencerow[IV]{#4}
                }
            }
        }

    \begin{scope}[on background layer]
        \path[
            fill=sequencecolor,
        ] 
            ([xshift=-0.75mm, yshift=1mm] topturn.north)
            --
            ([yshift=1.75mm] topturn.north)
            --
            ([xshift=0.75mm, yshift=1mm] topturn.north)
            --
            ([xshift=0.75mm, yshift=-1mm] turn.south)
            --
            ([yshift=-1.75mm] turn.south)
            --
            ([xshift=-0.75mm, yshift=-1mm] turn.south)
            --
            cycle;
    \end{scope}

    \end{tikzpicture}%
}

%%%


%%% Misc rendering helper functions

\newcommand{\rendermanasymbol}[2]{
    \directlua{
        function get_mana_filename(s)
            symbols = {}
            symbols["A"] = "any"
            symbols["F"] = "focus"
            symbols["S"] = "strength"
            symbols["W"] = "will"
            symbols["X"] = "spirit"
            return symbols[s] or s
        end
    }
    \tikz[overlay, remember picture] 
    \node[
        anchor=north west, 
        inner ysep=0pt,
        inner xsep=0pt,
        outer ysep=0pt,
        outer xsep=0pt,
        xshift=5.5mm,
        yshift=-\directlua{tex.print(5.5 + 6 * (#1 - 1))}mm,
    ] at (current page.north west) {
        % \includesvg[width = 63mm]{frames/frame-berserker.svg}
        \includegraphics[width=5mm]{icons/mana/\directlua{tex.print(get_mana_filename(\luastringN{#2}))}.png}
    };
}

\begin{luacode*}
    function render_mana(s)
        for i = 1, #s do
            tex.print("\\rendermanasymbol{"..i.."}{"..s:sub(i,i).."}")
        end
    end
\end{luacode*}
\newcommand{\rendermana}[1]{\directlua{render_mana(\luastring{#1})}}


% this one is for use inside the card text
\newcommand{\manatext}[1]{%
    \smash{\scalerel*{\includegraphics{icons/mana/#1.png}}{\strut}}%
    % \includegraphics[width=1mm]{icons/mana/#1.png}%
}
\begin{luacode*}
    function render_mana_text(s)
        for i = 1, #s do
            tex.sprint("\\manatext{"..get_mana_filename(s:sub(i,i)).."}")
            if i ~= #s then 
                -- the \, space is technically a kern, 
                -- which stops latex from breaking it with a newline
                tex.sprint("\\,")
            end
        end
    end
\end{luacode*}
\newcommand{\mana}[1]{\directlua{render_mana_text(\luastring{#1})}}


% building the attack and block icons
% generate gradient png
% magick -size 474x473 -define gradient:direction=Northeast gradient:"#00DB1D"-"#BF00FF" gradient.png

% two commands for the overlaying
% convert gradient.png attack.png -composite resulta.png
% convert resulta.png attack-mask-2.png  -compose copy_opacity -composite result.png

% and this just combines both of them into one
% convert gradient.png attack.png -composite attack-mask.png -compose copy_opacity -composite result.png

% and then this combines all three into one:
% convert -size 474x473 -define gradient:direction=Northeast gradient:"#00DB1D"-"#BF00FF" attack.png -composite attack-mask.png -compose copy_opacity -composite result.png


% better version that fixes the "colored edge" issue
% convert -size 474x473 -define gradient:direction=Northeast gradient:"#00DB1D"-"#BF00FF" attack-mask.png -compose copy_opacity -composite -channel a -threshold 99% attack.png -channel rgba -compose src-over -composite result.png

\begin{luacode*}
    function generate_icon(name, col1, col2)
        -- local file = io.popen("cd icons/ && convert -size 473x473 -define gradient:direction=Northeast gradient:\"#"..col1.."\"-\"#"..col2.."\" "..name..".png -composite "..name.."-mask.png -compose copy_opacity -composite "..name.."-"..col1.."-"..col2..".png")
        local file = io.popen("cd icons/ && convert -size 473x473 -define gradient:direction=Northeast gradient:\"#"..col1.."\"-\"#"..col2.."\" "..name.."-mask.png -compose copy_opacity -composite -channel a -threshold 99% "..name..".png -channel rgba -compose src-over -composite "..name.."-"..col1.."-"..col2..".png")
        local output = file:read('*all')
        file:close()
    end
    function get_icon(name, col1, col2)
        filename = "icons/"..name.."-"..col1.."-"..col2..".png"
        -- check if it already exists
        local f = io.open(filename, "r")
        if f~=nil then 
            io.close(f) 
            return filename
        else 
            generate_icon(name, col1, col2)
            return filename  
        end
    end
\end{luacode*}
\newcommand{\geticon}[3]{\directlua{tex.print(get_icon(\luastring{#1}, \luastring{#2}, \luastring{#3}))}}



% this one is for use inside the card text
\newcommand{\attackicon}{%
    \smash{\scalerel*{%
        \begin{tikzpicture}[x=5mm, y=5mm]
            \node[
                % anchor=east, 
                inner sep=0pt,
            ] (icon) {%
                \includegraphics{\geticon{attack}{\pgfkeysvalueof{/card/color-left}}{\pgfkeysvalueof{/card/color-right}}}%
            };
            \node[
                anchor=center, 
                inner sep=0pt,
                opacity = 0.5,
            ] at (icon.center) {%
                \includegraphics{\geticon{attack}{FFFFFF}{FFFFFF}}%
            };
        \end{tikzpicture}%
    }{\strut}}%
    % \includegraphics[width=1mm]{icons/mana/#1.png}%
}
