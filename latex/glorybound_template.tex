% !TeX program = lualatex

\documentclass{article}
% \documentclass{minimal}
\usepackage[
    paperwidth=63mm, 
    paperheight=88mm,
    margin=0mm,
]{geometry}
\setlength\parindent{0pt}
\setlength{\parskip}{2mm}

% the cmyk option makes it convert rgb color to cmyk space
% this makes the pdf output look more "realistic", but less vibrant
% \usepackage[cmyk]{xcolor}
\usepackage{xcolor}

% makes lua stuff easier
\usepackage{luacode}

% for \NewDocumentCommand
\usepackage{xparse}

% for filler text
\usepackage{lipsum}

% for getting the current date
\usepackage[datesep=/]{datetime2}
\DTMsetdatestyle{mmddyy}

% for textboxes
% this also pulls in tikz as a dependency
\usepackage[fitting, skins]{tcolorbox}
\usetikzlibrary{backgrounds}
\usetikzlibrary{fit}

% for length maths shenanigans
\usepackage{calc}

% for tables stuff
\usepackage{tabularx}
\usepackage{array}

% for outlined text
\usepackage{pdfrender}

% for scaling things
\usepackage{scalerel}

% this gets rid of a bunch of font scaling warnings 
% by replacing the standard fonts with scalable versions
\usepackage{lmodern}

% for better conditionals
\usepackage{ifthen}
\usepackage{etoolbox}

% for landscape pages
\usepackage{pdflscape}

% load fonts
\usepackage{fontspec}
\newfontfamily\bodyfont{CrimsonPro-Regular}[
    Path           = "fonts/Crimson_Pro/",
    Extension      = .ttf,
    Ligatures      = TeX,
    Numbers        = Monospaced,
    BoldFont       = CrimsonPro-Bold,
    ItalicFont     = CrimsonPro-Italic,
    BoldItalicFont = CrimsonPro-BoldItalic,
    FontFace       = {sb}{n}{CrimsonPro-SemiBold},
    FontFace       = {sb}{it}{CrimsonPro-SemiBoldItalic},
]
\newfontfamily\titlefont{Grenze-Regular}[
    Path           = "fonts/Grenze/",
    Extension      = .ttf,
    Ligatures      = TeX,
    BoldFont       = Grenze-Bold,
    ItalicFont     = Grenze-Italic,
    BoldItalicFont = Grenze-BoldItalic,
    FontFace       = {sb}{n}{Grenze-SemiBold},
    FontFace       = {sb}{it}{Grenze-SemiBoldItalic},
]

% semibold commands
\DeclareOldFontCommand{\sbseries}{\fontseries{sb}\selectfont}{\mathbf}
\DeclareTextFontCommand{\textsb}{\sbseries}




\begin{document}

% my custom textbox command
% Some tcolorbox hackery to add "proper" one-line vertical centering
\newlength{\Eheight}
\makeatletter
\def\tcb@dbox@top#1#2#3#4#5{\pgftext[x=#1,y=#2+#3,left,top]{#5}}
% define new valign type
\def\tcb@dbox@centerline#1#2#3#4#5{%
    % based on valign=top 
    \pgftext[x=#1,y=#2+#3,left,top]{%
        % make sure we have the correct front set for measuring 
        \pgfkeysvalueof{/textbox/font}%
        \fontsize{\tcbfitdim}{\tcbfitdim}\selectfont%
        % measure cap height
        \setlength{\Eheight}{\heightof{E}}%
        % \the\Eheight
        % and then raise (actually lower) the text based on that
        \raisebox{%
            0pt-0.5\Eheight-(\pgfkeysvalueof{/textbox/height}-\pgfkeysvalueof{/textbox/margin}-\pgfkeysvalueof{/textbox/margin})/2
        }{#5}
    }
}
\def\tcb@dbox@center#1#2#3#4#5{\pgftext[x=#1,y=#2+#3/2,left]{#5}}
\def\tcb@dbox@bottom#1#2#3#4#5{\pgftext[x=#1,y=#2,left,bottom]{#5}}

\tcbset{
    % add new valign type to the list of valid options
    valign/centerline/.code={\def\kvtcb@valignupper{centerline}}
}
\makeatother


\pgfkeys{/textbox/.cd,% set the initial path
    anchor/.initial=north,
    anchorpos/.initial=north,
    x/.initial=0mm,
    y/.initial=0mm,
    rotate/.initial=0,
    width/.initial=54mm,
    height/.initial=4mm,
    margin/.initial=1mm,
    fontsize/.initial=10pt,
    font/.initial=\bodyfont,
}

% first argument (optional): key-value options
% second argument (optional): extra stuff passed to tcolorbox
%third argument: textbox contents
\NewDocumentCommand{\bettertextbox}{ O{} O{} +m }{
    \begingroup%
    % handle key-value arguments
    \pgfkeys{/textbox/.cd, #1}%
% 
    \tikz[overlay, remember picture] 
    \node[
        anchor=\pgfkeysvalueof{/textbox/anchor}, 
        xshift=\pgfkeysvalueof{/textbox/x},
        yshift= -1 * \pgfkeysvalueof{/textbox/y},
        inner ysep=0pt,
        inner xsep=0pt,
        outer ysep=0pt,
        outer xsep=0pt,
        rotate=\pgfkeysvalueof{/textbox/rotate},
    ] at (current page.\pgfkeysvalueof{/textbox/anchorpos}) {%
        \begin{tcolorbox}[
            % show bounding box=green,
            width=\pgfkeysvalueof{/textbox/width},
            height=\pgfkeysvalueof{/textbox/height},
            fit,
            fit basedim=\pgfkeysvalueof{/textbox/fontsize},
            fit skip=1.1, % baselineskip
            blank,
            parbox=false, % makes parskip work
            halign=flush left,
            boxsep=\pgfkeysvalueof{/textbox/margin},
            % fontupper=\fontsize{\pgfkeysvalueof{/textbox/fontsize}}{\pgfkeysvalueof{/textbox/fontsize}}\selectfont,
            % tikz={opacity=0.5,transparency group},
            #2
        ]%%
            \color{dark}
            \pgfkeysvalueof{/textbox/font}%
            % this lets the parskip adapt to scaling text, 
            % it makes the whitespace look more natural 
            % and leaves more space for text
            \setlength{\parskip}{0.5\tcbfitdim}%
            #3
        \end{tcolorbox}%
    };%
    \endgroup%
}


% helpers for file splitting / output shenanigans
\begin{luacode*}
    start_page = 0
    end_page = 0
    file_name = nil
    -- the "w" open mode erases old content
    page_file = io.open("page-numbers-paths.log", "w") 
    card_file = io.open("page-numbers-cards.log", "w") 
    current_page = 0
\end{luacode*}


\newcommand{\countpage}{%
    \directlua{
        current_page = current_page + 1
    }
}

\newcommand{\registercard}[2]{%
    \directlua{
        path_name = \luastring{#1}
        card_name = \luastring{#2}
        card_file:write(path_name..' - '..card_name..' - '..current_page..'\string\n')
    }
}

% \newcommand{\beginpathfile}[1]{%
%     \directlua{
%         file_name = \luastring{#1}
%         start_page = current_page + 1
%     }
% }

% \newcommand{\finishpathfile}{%
%     \directlua{
%         end_page = current_page
%         if file_name then
%             page_file:write(file_name..' '..start_page..' '..end_page..'\string\n')
%         end
%     }
% }


% affinities stuff
\definecolor{noaffinity}{HTML}{606060}

% \definecolor{skulls}{HTML}{AF0008}
% \definecolor{eyes}{HTML}{00E3FF}
% \definecolor{daggers}{HTML}{220087}
% \definecolor{blossoms}{HTML}{0CB71D}
% \definecolor{candles}{HTML}{FF4300}
% \definecolor{claws}{HTML}{E77000}
% \definecolor{shields}{HTML}{FFC935}
% \definecolor{hours}{HTML}{D75BFF}
% \definecolor{masks}{HTML}{ED0898}
% \definecolor{ribbons}{HTML}{AFED77}

\definecolor{skulls}{HTML}{800006}
\definecolor{eyes}{HTML}{1992B0}
\definecolor{daggers}{HTML}{220087}
\definecolor{blossoms}{HTML}{265906}
\definecolor{candles}{HTML}{CC2C16}
\definecolor{claws}{HTML}{D66800}
\definecolor{shields}{HTML}{FFBB00}
\definecolor{hours}{HTML}{512961}
\definecolor{masks}{HTML}{FF57C1}
\definecolor{ribbons}{HTML}{8DEBA4}


% card rendering code
\colorlet{dark}{black}
\colorlet{light}{white}
% \colorlet{dark}{red}
% \colorlet{light}{green}
% \colorlet{dark}{white}
% \colorlet{light}{black!90}

\definecolor{heirloomlight}{HTML}{E3C9AD}


%%% Commands to be used in card definitions

\colorlet{ugradegray}{dark!25!light!75}
% \newcommand{\upgrade}[1]{\sethlcolor{ugradegray}\hl{{#1}}}
% \newcommand{\upgrade}[1]{\colorbox{ugradegray}{ugradegray}{#1}}

% \newcommand{\upgrade}[1]{%
% % #1%
%     \setlength{\fboxsep}{0pt}%
%     \colorbox{ugradegray}{%
%         % \vphantom{Blg}%
%         \strut%
%         #1%
%         % \makebox{something}%
%         % \makebox{
%         %     \strut#1
%         % }
%     }%
% }

\newcommand{\printsepifnotempty}[3]{%
    \directlua{
        s1 = \luastring{#1}
        sep = \luastring{#2}
        s3 = \luastring{#3}

        tex.sprint(s1)
        if s1:len() > 0 and s3:len() > 0 then
            tex.sprint(sep)
        end
        tex.sprint(s3)
    }%
}
\newcommand{\printsepifnotemptyor}[3]{%
    \directlua{
        s1 = \luastring{#1}
        sep = \luastring{#2}
        s3 = \luastring{#3}

        tex.sprint(s1)
        if s1:len() > 0 or s3:len() > 0 then
            tex.sprint(sep)
        end
        tex.sprint(s3)
    }%
}

\newcommand{\printseponlyifnotempty}[2]{%
    \directlua{
        sep = \luastring{#1}
        s = \luastring{#2}

        if s:len() > 0 then
            tex.sprint(sep)
        end
    }%
}

\newcommand{\printseponlyifempty}[2]{%
    \directlua{
        sep = \luastring{#1}
        s = \luastring{#2}

        if s:len() == 0 then
            tex.sprint(sep)
        end
    }%
}

% https://tex.stackexchange.com/a/225078
\newcommand\fauxsc[1]{\fauxschelper#1 \relax\relax}
\def\fauxschelper#1 #2\relax{%
  \fauxschelphelp#1\relax\relax%
  \if\relax#2\relax\else\ \fauxschelper#2\relax\fi%
}
\def\Hscale{.80}\def\Vscale{.80}\def\Cscale{1.00}
\def\fauxschelphelp#1#2\relax{%
  \ifnum`#1=\lccode`#1\relax\scalebox{\Hscale}[\Vscale]{\char\uccode`#1}\else%
    \scalebox{\Cscale}[1]{#1}\fi%
  \ifx\relax#2\relax\else\fauxschelphelp#2\relax\fi}

\begin{luacode*}
    function remove_apostrophe(path)
        new = path:gsub("'", "_")
        -- texio.write("TEST:"..new)
        return new
    end
\end{luacode*}
\newcommand{\fixpath}[1]{\directlua{tex.sprint(remove_apostrophe(\luastring{#1}))}}



\makeatletter
\newcommand{\upgrade}[2]{%
% #1%
    \setlength{\fboxsep}{0pt}%
    \colorbox{ugradegray}{%
        % \vphantom{Blg}%
        \strut%
        \,#1\,%
        % \makebox{something}%
        % \makebox{
        %     \strut#1
        % }
    }%
    \ifstrempty{#2}{}{%
        \ifx\upgradereplacement\empty%
        \else%
            \g@addto@macro{\upgradereplacement}{;\ }%
        \fi%
        \g@addto@macro{\upgradereplacement}{\{#2\}}%
    }
}
\makeatother

\newcommand{\dividerline}{%
    \vspace{-0.667\parskip}%
    \par%
    \centerline{%
        % \rule{.8\linewidth}{.125mm}%
        % \rule{36mm}{.125mm}%
        \rule{43.2mm}{.125mm}%
    }%
    \par%
    \vspace{-0.667\parskip}%
}%

% \newcommand{\onplay}[2][]{%
%     \textbf{On play\printseponlyifempty{:}{#1}}\printsepifnotemptyor{}{, }{#1}\printseponlyifnotempty{:}{#1} #2
%     \dividerline
% }

\newcommand{\thinparbreak}{{\setlength{\parskip}{0.5\parskip}\par}}

\NewDocumentCommand{\onplaynoline}{+o +m}{%
    {\textbf{On play\IfNoValueT{#1}{:}}\IfValueT{#1}{, #1} #2}
}

\NewDocumentCommand{\onplay}{+o +m}{%
    \onplaynoline[#1]{#2}%
    % \dividerline
}

% \newcommand{\upgrade}{\bgroup\markoverwith{\textcolor{ugradegray}{\rule[-.5ex]{.5pt}{2.5ex}}}\ULon}

% \newcommand{\block}{{\par\centering\textbf{Block} an attack.\par}}

% \newcommand{\attack}[1]{{\par\centering\textbf{Attack} #1.\par}}

\pgfkeys{/fancyrenderer/.cd,% set the initial path
    text/.initial={},
    icon/.initial={},
    icontext/.code={\pgfkeyssetvalue{/fancyrenderer/icontext}{#1}},
    fillcolor/.initial={light},
    textcolor/.initial={dark},
    scale/.initial={1.0},
    % scale/.initial={0.8},
}

% helper lengths
\newlength\boxwidth
\newlength\cornersize
\newlength\iconwidth
\newlength\tempfontsize
\newlength\outlinewidth

% https://tex.stackexchange.com/a/6424
\makeatletter
\newcommand*{\DivideLengths}[2]{%
  \strip@pt\dimexpr\number\numexpr\number\dimexpr#1\relax*65536/\number\dimexpr#2\relax\relax sp\relax
}
\makeatother

\NewDocumentCommand{\fancyrenderer}{ +O{} }{%
    % to make sure no arguments "leak" between different calls
    \begingroup%
    % some magic for autoscaling
    % \pgfkeyssetvalue{/fancyrenderer/scale}{\DivideLengths{\tcbfitdim}{\pgfkeysvalueof{/textbox/fontsize}}}%
    \pgfkeyssetvalue{/fancyrenderer/scale}{%
        \directlua{%
            ratio = \DivideLengths{\tcbfitdim}{\pgfkeysvalueof{/textbox/fontsize}}
            % tex.print(ratio)
            if ratio > 0.9 then
                tex.print("1.0")
            % elseif ratio < 0.7 then
            %     tex.print("0.6")
            else
                tex.print("0.8")
            end
        }
    }%
    \setlength\boxwidth{36mm}%
    % horizontal centering (relative to the parent parbox)
    \ifdefined\insidesequence
        \pgfkeyssetvalue{/fancyrenderer/scale}{0.8}%
        \setlength\boxwidth{32mm}%
    \else
        \centering%
    \fi
    % handle key-value arguments
    \pgfkeys{/fancyrenderer/.cd, #1}%
    \begin{tikzpicture}[x=1mm, y=1mm]
        \node[
            anchor=center,
            rectangle,
            % minimum width=0.8\linewidth, 
            minimum width=\boxwidth, 
            minimum height=6mm*\pgfkeysvalueof{/fancyrenderer/scale},
            % fill=gray, 
        ] (bb) {};


        \setlength\cornersize{1.5mm*\real{\pgfkeysvalueof{/fancyrenderer/scale}}}
        \path[
            draw=dark, line width=0.5mm, 
            fill=\pgfkeysvalueof{/fancyrenderer/fillcolor}
        ] 
            ([xshift=-2mm, yshift=-0.25mm] bb.north east)
            --
            ([xshift=\cornersize, yshift=-0.25mm] bb.north west)
            --
            ([xshift=0.25mm, yshift=-\cornersize] bb.north west)
            -- 
            ([xshift=0.25mm, yshift=\cornersize] bb.south west)
            --
            ([xshift=\cornersize, yshift=0.25mm] bb.south west)
            -- 
            ([xshift=-2mm, yshift=0.25mm] bb.south east)
            -- 
            cycle;

        \setlength\tempfontsize{10pt*\real{\pgfkeysvalueof{/fancyrenderer/scale}}}

        \node[
            anchor=west,
            % anchor=mid west,
            % minimum width=0.8\linewidth-8mm,
            minimum width=\boxwidth-8mm,
        ] at (bb.west) {
            \fontsize{\tempfontsize}{\tempfontsize}\selectfont
            \textcolor{\pgfkeysvalueof{/fancyrenderer/textcolor}}{\pgfkeysvalueof{/fancyrenderer/text}}
        };

        \setlength\iconwidth{10mm*\real{\pgfkeysvalueof{/fancyrenderer/scale}}}

        % attack icon
        \node[
            anchor=east, 
            inner ysep=0pt,
            inner xsep=0pt,
            outer ysep=0pt,
            outer xsep=0pt,
        ] (icon) at (bb.east) {%
            \includegraphics[width=\iconwidth]{\geticon{\pgfkeysvalueof{/fancyrenderer/icon}}{\pgfkeysvalueof{/card/color-left}}{\pgfkeysvalueof{/card/color-right}}}%
        };

        % and the white-ish semi-transparent overlay to make the colors a bit more wahsed out
        \node[
            anchor=east, 
            inner ysep=0pt,
            inner xsep=0pt,
            outer ysep=0pt,
            outer xsep=0pt,
            opacity = 0.5,
        ] at (bb.east) {%
            \pgfkeysifdefined{night}{%
                \includegraphics[width=\iconwidth]{\geticon{\pgfkeysvalueof{/fancyrenderer/icon}}{000000}{000000}}%
            }{%
                \includegraphics[width=\iconwidth]{\geticon{\pgfkeysvalueof{/fancyrenderer/icon}}{FFFFFF}{FFFFFF}}%
            }%
        };

        \setlength\tempfontsize{24pt*\real{\pgfkeysvalueof{/fancyrenderer/scale}}}
        \setlength\outlinewidth{0.75mm*\real{\pgfkeysvalueof{/fancyrenderer/scale}}}

        \pgfkeysifdefined{/fancyrenderer/icontext}{
            % https://tex.stackexchange.com/questions/400296/outline-text-characters
            \node[
                anchor=center,
                % minimum width=\iconwidth, 
                % minimum height=\iconwidth,
                % fill=gray,
            ] at (icon.center) {%
                \fontsize{\tempfontsize}{\tempfontsize}\selectfont
                \bfseries
                \textpdfrender{
                    TextRenderingMode=FillStrokeClip,
                    LineWidth=\outlinewidth,
                    FillColor=\pgfkeysifdefined{heirloom}{light!50!heirloomlight!50}{light},
                    StrokeColor=dark, 
                    MiterLimit=1,
                }{\pgfkeysvalueof{/fancyrenderer/icontext}}%
            };
            \node[
                anchor=center,
                % minimum width=\iconwidth, 
                % minimum height=\iconwidth,
            ] at (icon.center) {%
                \fontsize{\tempfontsize}{\tempfontsize}\selectfont
                \bfseries
                \textcolor{\pgfkeysifdefined{heirloom}{light!50!heirloomlight!50}{light}}{\pgfkeysvalueof{/fancyrenderer/icontext}}%
            };
        }{}

        % this just draws a little line over the left edge of the icon
        % to try and hide the transparancy issues there
            
        \ifthenelse{\equal{\directlua{
            icon = \luastring{\pgfkeysvalueof{/fancyrenderer/icon}}
            % texio.write('asdf'..icon)
            if icon:find('^block') == nil then
                tex.sprint('true')
            else
                tex.sprint('false')
            end
        }}{false}}{
            \path[
                draw=dark, line width=0.5mm, 
            ] 
                ([xshift=-\iconwidth*0.9, yshift=-0.0mm] bb.north east)
                --    
                ([xshift=-\iconwidth*0.9, yshift=-0.1mm] bb.north east)
                --    
                ([xshift=-\iconwidth] bb.east)
                --
                ([xshift=-\iconwidth*0.9, yshift=0.1mm] bb.south east)
                --
                ([xshift=-\iconwidth*0.9, yshift=0.0mm] bb.south east);
        }{}

        
    \end{tikzpicture}%
    \par
    \endgroup%
}

\newcommand{\block}{%
    \fancyrenderer[%
        text={\textbf{Block} an attack.},
        icon=\pgfkeysifdefined{night}{block-night}{block},
        fillcolor=dark,
        textcolor={\pgfkeysifdefined{heirloom}{heirloomlight}{light}},
    ]
}
\newcommand{\attack}[1]{%
    \fancyrenderer[%
        text={\textbf{Attack}},
        icon=\pgfkeysifdefined{night}{attack-night}{attack},
        icontext={#1},
        fillcolor={\pgfkeysifdefined{heirloom}{heirloomlight}{light}},
        textcolor=dark,
    ]
}

\newcommand{\innatebanner}{%
    \begingroup
    \centering
    \begin{tikzpicture}[x=1mm, y=1mm]
        \node[
            anchor=center, 
            inner sep=0pt
        ] {
            \includegraphics[height=5mm]{icons/innate-banner\pgfkeysifdefined{night}{-night}{}.png}
        };
    \end{tikzpicture}%
    \par
    \endgroup%
}

\newcommand{\oneshotbanner}{%
    \par
    \vspace*{0.5mm}
    \begingroup
    \centering
    \begin{tikzpicture}[x=1mm, y=1mm]
        \node[
            anchor=center, 
            inner sep=0pt
        ] {
            \includegraphics[height=5mm]{icons/oneshot-banner\pgfkeysifdefined{night}{-night}{}.png}
        };
    \end{tikzpicture}%
    \par
    \endgroup%
}




\newlength{\turncountersize}
\setlength{\turncountersize}{4.8mm}

\colorlet{sequencecolor}{dark!70}

\newcommand{\sequencerowsep}{%
    \node[
        anchor=north east,
        inner sep=0pt,
        minimum width=\linewidth, 
        minimum height=2mm,
    ] (end) at (text.south east) {};

    % \path[
    %     draw=sequencecolor, line width=0.25mm, 
    % ]
    % ([xshift=9mm] end.west) -- ([xshift=-9mm] end.east);
}
\newcommand{\sequencerow}[2][]{%
    % rules text
    \node[
        anchor=north east,
        % anchor=mid west,
        text width=\linewidth-6mm-3mm,
        minimum height=\turncountersize,
        inner sep=0pt,
        align=flush left,
        % fill=sequencecolor,
    ] (text) at (end.south east) {%
        \def\insidesequence{}%
        \fontsize{\tcbfitdim}{\tcbfitdim}\selectfont%
        % \color{light}
        #2
    };

    % turn counter 
    \node[
        fit={(text.north west) (text.south west)},
        anchor=east,
        rectangle,
        inner sep=0pt,
        minimum width=\turncountersize, 
        minimum height=\turncountersize,
        xshift=-3mm,
        % fill=sequencecolor,
    ] (turn) at (text.west) {};

    \path[
        draw=sequencecolor, line width=0.5mm, 
        fill=sequencecolor,
    ] 
        ([xshift=-1.2mm, yshift=-0.25mm] turn.north east)
        --
        ([xshift=1.2mm, yshift=-0.25mm] turn.north west)
        --
        ([xshift=0.25mm, yshift=-1.2mm] turn.north west)
        -- 
        ([xshift=0.25mm, yshift=1.2mm] turn.south west)
        --
        ([xshift=1.2mm, yshift=0.25mm] turn.south west)
        -- 
        ([xshift=-1.2mm, yshift=0.25mm] turn.south east)
        -- 
        ([xshift=-0.25mm, yshift=1.2mm] turn.south east)
        -- 
        ([xshift=-0.25mm, yshift=-1.2mm] turn.north east)
        --
        cycle;

    \node[
        anchor=center,
    ] at (turn.center) {%
        \fontsize{10pt}{10pt}\selectfont
        \textcolor{light}{\textbf{#1}}%
    };
}

\NewDocumentCommand{\sequence}{+o +o +o +o}{%
    \begin{tikzpicture}[x=1mm, y=1mm]
        \node[
            inner sep=0pt,
            minimum width=0mm, 
            minimum height=0mm,
            fill=sequencecolor, 
        ] (end) {};

        \sequencerow[I]{#1}
        % this node just saves the location of the first turn counter
        % so we can use it for drawing the timeline
        \node[
            anchor=north west,
            inner sep=0pt,
            minimum width=\turncountersize, 
            minimum height=0mm,
            fill=sequencecolor, 
            ] (topturn) at (turn.north west){};
        % add as many rows as we need
        \IfValueT{#2}
        {%
            \sequencerowsep
            \sequencerow[II]{#2}
            \IfValueT{#3}
            {%
            \sequencerowsep
                \sequencerow[III]{#3}
                \IfValueT{#4}
                {%
                    \sequencerowsep
                    \sequencerow[IV]{#4}
                }
            }
        }

    \begin{scope}[on background layer]
        \path[
            fill=sequencecolor,
        ] 
            ([xshift=-0.75mm, yshift=1mm] topturn.north)
            --
            ([yshift=1.75mm] topturn.north)
            --
            ([xshift=0.75mm, yshift=1mm] topturn.north)
            --
            ([xshift=0.75mm, yshift=-1mm] turn.south)
            --
            ([yshift=-1.75mm] turn.south)
            --
            ([xshift=-0.75mm, yshift=-1mm] turn.south)
            --
            cycle;
    \end{scope}

    \end{tikzpicture}%
}

%%%


%%% Misc rendering helper functions

\newcommand{\rendermanasymbol}[2]{
    \directlua{
        function get_mana_filename(s)
            symbols = {}
            symbols["A"] = "any"
            symbols["F"] = "focus"
            symbols["S"] = "strength"
            symbols["W"] = "will"
            symbols["X"] = "spirit"
            return symbols[s] or s
        end
    }
    \tikz[overlay, remember picture] 
    \node[
        anchor=north west, 
        inner ysep=0pt,
        inner xsep=0pt,
        outer ysep=0pt,
        outer xsep=0pt,
        xshift=5.5mm,
        yshift=-\directlua{tex.print(5.5 + 6 * (#1 - 1))}mm,
    ] at (current page.north west) {
        % \includesvg[width = 63mm]{frames/frame-berserker.svg}
        \includegraphics[width=5mm]{icons/mana/\directlua{tex.print(get_mana_filename(\luastringN{#2}))}.png}
    };
}

\begin{luacode*}
    function render_mana(s)
        for i = 1, #s do
            tex.print("\\rendermanasymbol{"..i.."}{"..s:sub(i,i).."}")
        end
    end
\end{luacode*}
\newcommand{\rendermana}[1]{\directlua{render_mana(\luastring{#1})}}


% this one is for use inside the card text
\newcommand{\manatext}[1]{%
    \smash{\scalerel*{\includegraphics{icons/mana/#1.png}}{\strut}}%
    % \includegraphics[width=1mm]{icons/mana/#1.png}%
}
\begin{luacode*}
    function render_mana_text(s)
        for i = 1, #s do
            tex.sprint("\\manatext{"..get_mana_filename(s:sub(i,i)).."}")
            if i ~= #s then 
                -- the \, space is technically a kern, 
                -- which stops latex from breaking it with a newline
                tex.sprint("\\,")
            end
        end
    end
\end{luacode*}
\newcommand{\mana}[1]{\directlua{render_mana_text(\luastring{#1})}}


% building the attack and block icons
% generate gradient png
% magick -size 474x473 -define gradient:direction=Northeast gradient:"#00DB1D"-"#BF00FF" gradient.png

% two commands for the overlaying
% convert gradient.png attack.png -composite resulta.png
% convert resulta.png attack-mask-2.png  -compose copy_opacity -composite result.png

% and this just combines both of them into one
% convert gradient.png attack.png -composite attack-mask.png -compose copy_opacity -composite result.png

% and then this combines all three into one:
% convert -size 474x473 -define gradient:direction=Northeast gradient:"#00DB1D"-"#BF00FF" attack.png -composite attack-mask.png -compose copy_opacity -composite result.png


% better version that fixes the "colored edge" issue
% convert -size 474x473 -define gradient:direction=Northeast gradient:"#00DB1D"-"#BF00FF" attack-mask.png -compose copy_opacity -composite -channel a -threshold 99% attack.png -channel rgba -compose src-over -composite result.png

\begin{luacode*}
    function generate_icon(name, col1, col2)
        -- local file = io.popen("cd icons/ && convert -size 473x473 -define gradient:direction=Northeast gradient:\"#"..col1.."\"-\"#"..col2.."\" "..name..".png -composite "..name.."-mask.png -compose copy_opacity -composite "..name.."-"..col1.."-"..col2..".png")
        local file = io.popen("mkdir -p icons/processed && cd icons/ && convert -size 473x473 -define gradient:direction=Northeast gradient:\"#"..col1.."\"-\"#"..col2.."\" "..name.."-mask.png -compose copy_opacity -composite -channel a -threshold 99% "..name..".png -channel rgba -compose src-over -composite processed/"..name.."-"..col1.."-"..col2..".png")
        local output = file:read('*all')
        file:close()
    end
    function get_icon(name, col1, col2)
        filename = "icons/processed/"..name.."-"..col1.."-"..col2..".png"
        -- check if it already exists
        local f = io.open(filename, "r")
        if f~=nil then 
            io.close(f) 
            return filename
        else 
            generate_icon(name, col1, col2)
            return filename  
        end
    end
\end{luacode*}
\newcommand{\geticon}[3]{\directlua{tex.print(get_icon(\luastring{#1}, \luastring{#2}, \luastring{#3}))}}



% this one is for use inside the card text
\newcommand{\attackicon}{%
    \smash{\scalerel*{%
        \begin{tikzpicture}[x=5mm, y=5mm]
            \node[
                % anchor=east, 
                inner sep=0pt,
            ] (icon) {%
                \includegraphics{\geticon{attack}{\pgfkeysvalueof{/card/color-left}}{\pgfkeysvalueof{/card/color-right}}}%
            };
            \node[
                anchor=center, 
                inner sep=0pt,
                opacity = 0.5,
            ] at (icon.center) {%
                \includegraphics{\geticon{attack}{FFFFFF}{FFFFFF}}%
            };
        \end{tikzpicture}%
    }{\strut}}%
    % \includegraphics[width=1mm]{icons/mana/#1.png}%
}


\newcommand{\upgradeicon}{%
    \smash{\scalerel*{%
            \includegraphics{icons/upgrade.png}%
    }{\strut}}%
}


\pgfkeys{/card/.cd,% set the initial path
    name/.initial={},
    % type/.initial={},
    cost/.initial={},
    text/.initial={},
    path/.initial={},
    color-left/.initial=FF00FF,
    color-right/.initial=0000FF,
    % types
    permanent/.code={\pgfkeyssetvalue{permanent}{#1}},
    sequence/.code={\pgfkeyssetvalue{sequence}{#1}},
    oneshot/.code={\pgfkeyssetvalue{oneshot}{#1}},
    innate/.code={\pgfkeyssetvalue{innate}{#1}},
    linked/.code={\pgfkeyssetvalue{linked}{#1}},
    linked type/.initial={},
    heirloom/.code={\pgfkeyssetvalue{heirloom}{#1}},
    support/.code={\pgfkeyssetvalue{support}{#1}},
    % purchase stuff
    purchase/.code={\pgfkeyssetvalue{purchase}{#1}},
    % upgrade stuff
    upgrade/.code={\pgfkeyssetvalue{upgrade}{#1}},
    upgrade cost/.code={\pgfkeyssetvalue{upgrade cost}{#1}},
    % read-only key to get the printed type line
    type/.initial={%
        \printsepifnotempty{%
            \pgfkeysifdefined{linked}{Linked}{%
                \pgfkeysifdefined{heirloom}{Heirloom}{%
                    \pgfkeysifdefined{purchase}{Advanced}{Starter}%
                }%
            }%
        }
        {\ --\ }
        {%
            \printsepifnotemptyor{\pgfkeysifdefined{innate}{Innate}{}}{ }{%
                \printsepifnotemptyor{\pgfkeysifdefined{support}{Support}{}}{ }{%
                    \printsepifnotemptyor{\pgfkeysifdefined{permanent}{Permanent}{}}{ }{%
                        \pgfkeysifdefined{sequence}{Sequence}{}%
                    }%
                }%
            }%
            % % supertypes
            % \pgfkeysifdefined{innate}{Innate }{}%
            % \pgfkeysifdefined{support}{Support }{}%
            % % \pgfkeysifdefined{linked}{Linked }{}%
            % % base types
            % \pgfkeysifdefined{permanent}{Permanent }{}%
            % \pgfkeysifdefined{sequence}{Sequence }{}%
        }
    },
    % art path
    art path/.initial={%
        art/\pgfkeysvalueof{/card/path}/\fixpath{\pgfkeysvalueof{/card/name}}.png%
    },
    % night mode toggle, not meant to be used directly in card definitions
    night/.code={\pgfkeyssetvalue{night}{#1}},%
    big art/.code={\pgfkeyssetvalue{big art}{#1}},%
    designer/.code={\pgfkeyssetvalue{designer}{#1}},%
}

\newcommand{\setpath}[3]{
    % \finishpathfile
    \pgfkeys{/card/.cd,
        path=#1,
        color-left=#2,
        color-right=#3,
    }
    % \beginpathfile{#1}
}

\newsavebox{\rulestextbox}
\newlength{\savedfontsize}
\newlength{\savedheight}
\newlength{\extraheight}
\newcommand{\buildbox}[1][0mm]{}

\newlength{\textboxbottomoffset}
\setlength{\textboxbottomoffset}{0mm}

\newlength{\templen}

\makeatletter
\let\upgradereplacement\@empty
\makeatother

% to trim art images to the size of the content:
% mogrify -trim +repage *.png

% to fill art with the path's gradient:
% convert -size $(identify -format %Wx%H Retribution.png) -define gradient:direction=Northeast gradient:"#00DB1D"-"#BF00FF" Retribution.png -compose copy_opacity -composite result.png

\begin{luacode*}
    function generate_art(folder, name, col1, col2)
        local file = io.popen("cd '"..folder.."' && convert -size $(identify -format %Wx%H '"..name..".png') -define gradient:direction=Northeast gradient:'#"..col1.."'-'#"..col2.."' '"..name..".png' -compose copy_opacity -composite '"..name.."-"..col1.."-"..col2..".png'")
        local output = file:read('*all')
        file:close()
    end
    function get_art(path, col1, col2)
        name = path:match("^.*/([^/]*).png$")
        folder = path:match("^(.*)/[^/]*$")
        filename = folder.."/"..name.."-"..col1.."-"..col2..".png"
        texio.write("artname: "..filename)
        -- check if it already exists
        local f = io.open(""..filename.."", "r")
        if f~=nil then 
            io.close(f) 
            return filename
        else 
            generate_art(folder, name, col1, col2)
            return filename  
        end
    end
\end{luacode*}
\newcommand{\getart}[3]{\directlua{tex.print(get_art(\luastring{#1}, \luastring{#2}, \luastring{#3}))}}

\makeatletter
\NewDocumentCommand{\card}{ +O{} }{

    % start a new page if we have to
    \clearpage

    % to make sure no arguments "leak" between different \card calls
    \begingroup
    % handle key-value arguments
    \pgfkeys{/card/.cd, #1}
    
    % increase our page counter and track card
    \countpage
    \registercard{\pgfkeysvalueof{/card/path}}{\pgfkeysvalueof{/card/name}}

    % handle night-mode cards
    % \pgfkeyssetvalue{night}{}%
    \pgfkeysifdefined{linked}{%
        \pgfkeyssetvalue{night}{}%
    }{}
    
    % set night mode colors
    \pgfkeysifdefined{night}{%
        % \colorlet{dark}{red}
        % \colorlet{light}{green}
        \colorlet{dark}{white}
        \colorlet{light}{black!90}
        \colorlet{sequencecolor}{dark!70!light!30}
        \colorlet{ugradegray}{dark!25!light!75}
    }{%
        \colorlet{dark}{black}
        \colorlet{light}{white}
        \colorlet{sequencecolor}{dark!70}
        \colorlet{ugradegray}{dark!25!light!75}
    }

    % render frame
    % the tikz stuff makes sure it's rendered in the background and not
    % as part of the document flow
    
    % background gradient:
    \definecolor{left}  {HTML}{\pgfkeysvalueof{/card/color-left}}
    \definecolor{right} {HTML}{\pgfkeysvalueof{/card/color-right}}
    \tikz[overlay, remember picture] 
    \node[
        anchor=north,
        yshift=-1.5mm,
        rectangle, 
        minimum width=\paperwidth-3mm, 
        minimum height=\paperheight-3mm,
        left color=left, 
        right color=right,
        shading = axis,
        shading angle=135, 
    ] at (current page.north) {};
    
    % frame (with transparency):
    \tikz[overlay, remember picture] 
    \node[
        anchor=north, 
        inner ysep=0pt,
        inner xsep=0pt,
        outer ysep=0pt,
        outer xsep=0pt,
    ] at (current page.north) {
        % \includegraphics[width=63mm]{icons/frame2.png}
        \includegraphics[width=63mm]{%
            % \pgfkeysifdefined{linked}{icons/frame2-gray.png}{icons/frame2.png}
            % icons/frame2.png
            % icons/frame2\pgfkeysifdefined{permanent}{-permanent}{}\pgfkeysifdefined{night}{-night}{}.png
            icons/frame2\pgfkeysifdefined{heirloom}{-heirloom}{}\pgfkeysifdefined{permanent}{-permanent}{}\pgfkeysifdefined{night}{-night}{}.png
            % icons/frame2-dark.png
        }
    };

    % cost
    \rendermana{\pgfkeysvalueof{/card/cost}}
    % \rendermanasymbol{1}{W}
    % \rendermanasymbol{2}{S}
    % \rendermanasymbol{3}{A}
    
    % make sure we have the right font ny default
    \bodyfont
    
    % name
    \bettertextbox
        [y=4.5mm, width=36mm, height=6mm, margin=0mm, fontsize=14pt, font=\titlefont\bfseries]
        [halign=center, valign=centerline]
    {%
        \pgfkeysvalueof{/card/name}%
    }
     
    % type
    \bettertextbox
        [y=11mm, width=36mm, height=4mm, margin=0mm, fontsize=8pt, font=\titlefont\sbseries\itshape]
        [halign=center, valign=centerline]
    {%
        \pgfkeysvalueof{/card/type}%
        % Basic -- \pgfkeysvalueof{/card/type}
    }

    % purchase cost?
    \newcommand{\purchaseunderlay}{%
    \begin{tcbclipinterior}
        \node[
            fit={(frame.north west) (frame.south east)},
            anchor=center,
            rectangle,
            rounded corners=1mm,
            inner sep=0pt,
            fill=dark!25!light!75,
        ] at (frame.center) {};
    \end{tcbclipinterior}
    }
    \pgfkeysifdefined{purchase}{%
        \bettertextbox
            [%
                y=5mm, x=26.5mm,
                width=6mm, height=10.5mm, margin=1mm, 
                anchor=north east,
                fontsize=14pt, 
                font=\bfseries,
            ]
            [%
                halign=center, valign=center,
                underlay={\purchaseunderlay},
                underlay={
                    \begin{tcbclipframe}
                        \node[
                            anchor=north,
                            inner sep=0pt,
                            yshift=-1mm,
                            % opacity=0.5,
                        ] at (frame.north) {\includegraphics[height=3.5mm]{icons/padlock\pgfkeysifdefined{night}{-night}{}.png}};
                    \end{tcbclipframe}
                },
                top=5.5mm,
            ]
        {%
            \pgfkeysvalueof{purchase}%
        }
    }{%
        % not defined
    }

    % upgrade?
    \setlength{\extraheight}{0mm}
    \newcommand{\upgradeunderlay}{%
        \begin{tcbclipinterior}
            \node[
                fit={(frame.north west) (frame.south east)},
                anchor=center,
                rectangle,
                rounded corners=1mm,
                inner sep=0pt,
                fill=\pgfkeysifdefined{heirloom}{dark!25!heirloomlight!75}{dark!25!light!75},
            ] at (frame.center) {};
        \end{tcbclipinterior}
    }
    \pgfkeysifdefined{upgrade cost}{%
        \bettertextbox
            [%
                y=81.5mm, x=27mm,
                width=42mm, height=6mm, margin=1mm, 
                anchor=south east,
                fontsize=8pt,
            ]
            [%
                halign=flush center, valign=center, left=1mm, right=1mm,
                fit height plus=4mm,
                underlay={%
                    \upgradeunderlay%
                    \setlength{\extraheight}{\tcbtextheight}
                    \global\extraheight=\extraheight
                },
            ]
        {%
            % this is purely to "evaluate" /card/text,
            % so we can force all the \upgrade{}{} commands to be executed
            \global\let\upgradereplacement\@empty%
            \sbox\rulestextbox{\parbox{54mm}{\pgfkeysvalueof{/card/text}}}%
            \upgradereplacement%
            \pgfkeysvalueof{upgrade}%
        }
        % add the margin back in
        \setlength{\extraheight}{\extraheight + 2mm}
        % \the\extraheight

        \setlength{\templen}{%
            \directlua{%
                if string.len(\luastring{\pgfkeysvalueof{upgrade cost}}) > 1 then
                    tex.sprint("12pt")
                else
                    tex.sprint("14pt")
                end
            }
        }
        \bettertextbox
            [%
                y=81.5mm, x=-27mm,
                width=11mm, height=\extraheight, margin=1mm, 
                anchor=south west,
                fontsize=\templen,
                font=\bfseries,
            ]
            [%
                halign=center, valign=centerline,
                underlay={\upgradeunderlay},
                underlay={
                    \begin{tcbclipframe}
                        \node[
                            anchor=west,
                            inner sep=0pt,
                            xshift=1mm,
                            % opacity=0.5,
                        ] at (frame.west) {\includegraphics[height=4mm]{icons/upgrade\pgfkeysifdefined{night}{-night}{}.png}};
                    \end{tcbclipframe}
                },
                left=4.5mm,
            ]
        {%
            \pgfkeysvalueof{upgrade cost}%
        }
        \addtolength{\textboxbottomoffset}{\extraheight}
    }{%
        % not defined
    }

    % oneshot?
    % \pgfkeysifdefined{oneshot}{%
    %     % defined 
    %     \tikz[overlay, remember picture] 
    %     \node[
    %         anchor=south, 
    %         inner ysep=0pt,
    %         inner xsep=0pt,
    %         outer ysep=0pt,
    %         outer xsep=0pt,
    %         % yshift=7.5mm,
    %         yshift=7mm+\textboxbottomoffset,
    %     ] at (current page.south) {
    %         % \includegraphics[width=36mm]{icons/oneshot-banner.png}
    %         \includegraphics[width=30mm]{icons/oneshot-banner\pgfkeysifdefined{night}{-night}{}.png}
    %     };
    %     \addtolength{\textboxbottomoffset}{5mm}
    % }{%
    %     % not defined
    % }

    % text
    \renewcommand{\buildbox}{%
        \sbox{\rulestextbox}{
            \setlength{\savedheight}{34mm - \textboxbottomoffset + \extraheight}
            \bettertextbox
                [%
                    y={81.5mm+\textboxbottomoffset}, 
                    width=54mm, 
                    % height=\pgfkeysifdefined{oneshot}{24mm}{32mm}, 
                    height=\savedheight, 
                    anchor=south, 
                    margin=2mm,
                ]
                [%
                    valign=center, %bottom=0.25mm, % the bottom=... just adds some extra spacing there
                    % top=-2mm,
                    top=\pgfkeysifdefined{innate}{-1mm}{0mm},
                    bottom=\pgfkeysifdefined{oneshot}{-1mm}{0.25mm},
                    underlay=\pgfkeysifdefined{permanent}{%
                        % \begin{tcbclipinterior}
                        %     \node[
                        %         anchor=center, 
                        %         opacity=0.15,
                        %         % yshift=1mm
                        %     ] at (frame.center) {
                        %         \includegraphics[height=14mm]{permanent-watermark.png}
                        %     };
                        % \end{tcbclipinterior}
                    }{}
                ] 
            {%
                \pgfkeysifdefined{innate}{%
                    % {\centering\includegraphics[width=36mm]{icons/innate-banner.png}\par}
                    % \textit{\textsb{Innate} (Play before the match starts.)}\par%
                    \innatebanner
                    % \textbf{Innate:} \textit{Play before the match starts.}\par%
                }{}%
                \pgfkeysifdefined{linked}{%
                    \textit{\printsepifnotemptyor{%
                        \pgfkeysvalueof{/card/linked type}%
                    }{\ --\ }{}%
                    \textsb{Linked} (This card begins the match banished and is recalled by \pgfkeysvalueof{linked}.)%
                    }\par%
                }{}%
                \pgfkeysifdefined{support}{%
                \textit{\textsb{Support} (Not used during match.)}%
                    \par%
                }{}%
                \pgfkeysvalueof{/card/text}%
                \global\setlength{\savedfontsize}{\tcbfitdim}
                \global\savedfontsize=\savedfontsize
                \pgfkeysifdefined{oneshot}{%
                    \oneshotbanner
                }{}%

            }%
        }
    }
    \setlength{\extraheight}{0mm}
    \pgfkeysifdefined{big art}{
        \setlength{\extraheight}{-14mm}
    }{}
    \buildbox
    \loop
    \ifdim\savedfontsize<8pt
        \setlength{\extraheight}{\extraheight + 2mm}
        \global\extraheight=\extraheight
        \buildbox
    \repeat
    % \ifdim\savedfontsize<7pt
    %     \setlength{\extraheight}{6mm}
    %     \buildbox
    % \else
    % \fi

    \usebox{\rulestextbox}
    % \the\savedfontsize

    \pgfkeysifdefined{designer}{%
        \bettertextbox[%
            % y={17mm+15mm-0.5\extraheight},
            y={32mm-0.5\extraheight},
            x=26.5mm,
            width={30mm-\extraheight}, height=6mm, margin=1mm, 
            anchor=north,
            rotate=-90,
            fontsize=5pt, 
            font=\titlefont,
        ]
        [%
            halign=center, valign=center,
        ]{%
            \textit{Champion's Design Challenge 2022\linebreak
            Designed by \textsb{\pgfkeysvalueof{designer}}}
        }
    }{}%
    

    % art:
    % (just a placeholder rectangle currently)
    % \typeout{ART: \pgfkeysvalueof{/card/art path}}
    \IfFileExists{\pgfkeysvalueof{/card/art path}}{
        \tikz[overlay, remember picture] 
        \node[
            anchor=north,
            yshift=-17mm,
            rectangle, 
            minimum width=36mm, 
            minimum height=30mm-\extraheight,
            inner sep=0pt,
            % fill=gray,
        ] at (current page.north) {%
            \includegraphics
                [width=36mm, height={30mm-\extraheight}, keepaspectratio]
                {%
                    \pgfkeysifdefined{heirloom}{%
                        \getart{\pgfkeysvalueof{/card/art path}}{000000}{000000}
                    }{%
                        \getart{\pgfkeysvalueof{/card/art path}}{\pgfkeysvalueof{/card/color-left}}{\pgfkeysvalueof{/card/color-right}}%
                    }%
                }%
        };
    }{%
        \tikz[overlay, remember picture] 
        \node[
            anchor=north,
            yshift=-17mm,
            rectangle, 
            minimum width=36mm, 
            minimum height=30mm-\extraheight,
            left color=left, 
            right color=right,
            shading = axis,
            shading angle=135, 
            opacity = 0.3,
        ] at (current page.north) {%
        };
    }
    
    % path
    % \pgfkeysifdefined{heirloom}{}{%
    \bettertextbox
        [y=81.5mm, x=-27mm, width=24mm, height=4mm, anchor=north west, font=\titlefont\color{white}\sbseries, fontsize=7pt, margin=0.5mm]
        [valign=centerline, left=-0.5mm]
    {%
        \pgfkeysifdefined{heirloom}{%
            \pgfkeysifdefined{designer}{%
                % \textit{Champion's Design Challenge}\linebreak
                % Designed by \pgfkeysvalueof{designer}%
            }{}%
            % \pgfkeysifdefined{designer}{2021 Champion Designer: \pgfkeysvalueof{designer}}{}%
        }{%
            \pgfkeysvalueof{/card/path}%
        }%
    }
    % }%

    
    % date - glorybound
    \bettertextbox
        [y=81.5mm, x=27mm, width=24mm, height=4mm, anchor=north east, font=\titlefont\color{white}\sbseries, fontsize=7pt, margin=0.5mm]
        [valign=centerline, right=-0.5mm, halign=flush right]
    {%
        % \today\ \ --\ \ Glorybound
        \today\ \ --\ \ GLORYBOUND
        % \today\ \ --\ \ \fauxsc{Glorybound}
    }

    \endgroup
}
\makeatother

\pgfkeys{/pathrow/.cd,% set the initial path
    name/.initial={},
    linked/.initial={},
    purchase/.code={\pgfkeyssetvalue{purchase}{#1}},
    upgrade/.code={\pgfkeyssetvalue{upgrade}{#1}},
}

\newcommand{\renderthingy}[4]{%
    \begingroup
    \newcommand{\purchaseunderlay}{%
        \begin{tcbclipinterior}
            \node[
                fit={(frame.north west) (frame.south east)},
                anchor=center,
                rectangle,
                rounded corners=1mm,
                inner sep=0pt,
                % this is some weird magic to get the border to be inside the shape
                % this does mean that the line width is being halved
                % https://tex.stackexchange.com/questions/245141/pgf-tikz-draw-the-border-of-a-shape-inside-it
                preaction={clip,postaction={fill=dark!25!light!75, draw=dark, line width=0.5mm}},
                % fill=dark!25!light!75,
                % draw=dark,
                % line width=0.25mm,
            ] at (frame.center) {};
        \end{tcbclipinterior}
    }
    \bettertextbox
        [%
            y={#3}, x={#2},
            width=6mm, height=10mm, margin=1mm, 
            anchor=center,
            fontsize=14pt, 
            font=\bfseries,
        ]
        [%
            halign=center, valign=center,
            underlay={\purchaseunderlay},
            underlay={
                \begin{tcbclipframe}
                    \node[
                        anchor=north,
                        inner sep=0pt,
                        yshift=-1mm,
                        % opacity=0.5,
                    ] at (frame.north) {\includegraphics[height=3.5mm]{#4}};
                \end{tcbclipframe}
            },
            top=5mm,
        ]
    {%
        #1%
    }
    \endgroup
}

\newlength{\yoffset}

\NewDocumentCommand{\pathrow}{ +O{} }{%
    % to make sure no arguments "leak" between different \card calls
    \begingroup%
    % handle key-value arguments
    \pgfkeys{/pathrow/.cd, #1}%
    %
    \setlength{\templen}{75.5mm-\yoffset}%
    %
    % purchase cost?
    \pgfkeysifdefined{purchase}{%
        \renderthingy{\pgfkeysvalueof{purchase}}{-23mm}{\templen}{icons/padlock.png}
    }{%
        \bettertextbox[%
            anchor=center,
            x=-23mm, y={\templen},
            width=7mm, height=2mm, margin=0mm,
            fontsize=7pt, font=\titlefont\sbseries,
            rotate=90,
        ]
        [halign=center, valign=centerline]
        {Starter}%
    }%
    %
    % upgrade cost?
    \pgfkeysifdefined{upgrade}{%
        \renderthingy{\pgfkeysvalueof{upgrade}}{23mm}{\templen}{icons/upgrade.png}
    }{}%
    %
    % card name
    \bettertextbox[%
        anchor=center,
        y={\templen}, 
        width=40mm, height=7mm, margin=1mm, 
        fontsize=12pt, font=\titlefont,
    ]
    [valign=centerline]
    {%
        \pgfkeysvalueof{/pathrow/name}%
    }%
    %
    \endgroup%
}


\pgfkeys{/path/.cd,% set the initial path
    name/.initial={},
    resources/.initial={},
    color-left/.initial=FF00FF,
    color-right/.initial=0000FF,
    % purchase/.code={\pgfkeyssetvalue{purchase}{#1}},
    % upgrade/.code={\pgfkeyssetvalue{upgrade}{#1}},
}

\NewDocumentCommand{\pathcard}{ +O{} +m }{

    % start a new page if we have to
    \clearpage

    % to make sure no arguments "leak" between different \card calls
    \begingroup
    % handle key-value arguments
    \pgfkeys{/path/.cd, #1}
    
    % increase our page counter and track card
    \countpage
    \registercard{\pgfkeysvalueof{/path/name}}{\pgfkeysvalueof{/path/name}}

    \colorlet{dark}{black}
    \colorlet{light}{white}
    \colorlet{sequencecolor}{dark!70}
    \colorlet{ugradegray}{dark!25!light!75}

    % render frame
    % the tikz stuff makes sure it's rendered in the background and not
    % as part of the document flow
    
    % background gradient:
    \definecolor{left}  {HTML}{\pgfkeysvalueof{/path/color-left}}
    \definecolor{right} {HTML}{\pgfkeysvalueof{/path/color-right}}
    \tikz[overlay, remember picture] 
    \node[
        anchor=north,
        yshift=-1.5mm,
        rectangle, 
        minimum width=\paperwidth-3mm, 
        minimum height=\paperheight-3mm,
        left color=left, 
        right color=right,
        shading = axis,
        shading angle=135, 
    ] at (current page.north) {};
    
    % frame (with transparency):
    \tikz[overlay, remember picture] 
    \node[
        anchor=north, 
        inner ysep=0pt,
        inner xsep=0pt,
        outer ysep=0pt,
        outer xsep=0pt,
    ] at (current page.north) {
        % \includegraphics[width=63mm]{icons/frame2.png}
        \includegraphics[width=63mm]{%
            icons/frame2-path.png
        }
    };

    % cost
    % \rendermana{\pgfkeysvalueof{/path/resources}}
    % \rendermanasymbol{1}{W}
    % \rendermanasymbol{2}{S}
    % \rendermanasymbol{3}{A}
    
    % make sure we have the right font by default
    \bodyfont
    
    % name
    \bettertextbox
        [y=8mm, width=36mm, height=7mm, margin=0mm, fontsize=18pt, font=\titlefont\bfseries]
        [halign=center, valign=centerline]
    {%
        \pgfkeysvalueof{/path/name}
    }
     
    % "path of the"
    \bettertextbox
        [y=5mm, width=36mm, height=4mm, margin=0mm, fontsize=8pt, font=\titlefont\sbseries]
        [halign=center, valign=centerline]
    {Path of the}

    \setlength{\yoffset}{0mm}

    % render rows
    #2%

    % path
    \bettertextbox
        [y=81.5mm, x=-27mm, width=24mm, height=4mm, anchor=north west, font=\titlefont\color{white}\sbseries, fontsize=7pt, margin=0.5mm]
        [valign=centerline, left=-0.5mm]
    {%
        \pgfkeysvalueof{/path/name}
    }
    
    % date - glorybound
    \bettertextbox
        [y=81.5mm, x=27mm, width=24mm, height=4mm, anchor=north east, font=\titlefont\color{white}\sbseries, fontsize=7pt, margin=0.5mm]
        [valign=centerline, right=-0.5mm, halign=flush right]
    {%
        \today\ \ --\ \ Glorybound
    }

    \endgroup
}


\BLOCK{ for path in paths }
    % \setpath{\VAR{path.name}}{\VAR{path.colors[0]}}{\VAR{path.colors[1]}}
    \setpath{\VAR{path.name}}
    \BLOCK{ if path.extras != None }
    \VAR{path.extras}
    \BLOCK{ endif }

    \pathcard[%
        name={\VAR{path.name}},
        color-left=\VAR{path.colors[0]},
        color-right=\VAR{path.colors[1]},
        \BLOCK{ if path.passive_name != None }
        passive name={\VAR{path.passive_name}},
        \BLOCK{ endif }
        \BLOCK{ if path.passive != None }
        passive={\VAR{path.passive}},
        \BLOCK{ endif }
        \BLOCK{ if path.primary != None }
        primary={\VAR{path.primary}},
        \BLOCK{ endif }
        \BLOCK{ if path.secondary != None }
        secondary={\VAR{path.secondary}},
        \BLOCK{ endif }
    ]{%
    \BLOCK{ if path.name != "Heirloom" }
        \BLOCK{ if path.cards|selectattr("linked", "eq", None)|list|last|attr("linked_to")|length > 0 }
            \addtolength{\yoffset}{1.5mm}%
        \BLOCK{ endif }
        \BLOCK{ for card in path.cards|reverse }
            \BLOCK{ if card.linked == None }
            \pathrow[%
                name={\VAR{card.name}},
                \BLOCK{ if card.purchase != None }
                purchase={\VAR{card.purchase}},
                \BLOCK{ endif }
                \BLOCK{ if card.upgrade_cost != None }
                upgrade={\VAR{card.upgrade_cost}},
                \BLOCK{ endif }
                \BLOCK{ if card.linked_to|length > 0 }
                linked={\VAR{card.linked_to|map(attribute='path_card_name')|join('\ \ --\ \ ')}},
                \BLOCK{ endif }
            ]%
            \addtolength{\yoffset}{12mm}%
            \BLOCK{ endif }
        \BLOCK{ endfor }
    \BLOCK{ endif }
    }

    \BLOCK{ if true }
        \BLOCK{ for card in path.cards }
            \BLOCK{ if true }
                \card[%
                    name={\VAR{card.name}},
                    cost={\VAR{card.cost}},
                    \BLOCK{ for type in card.types }
                    \VAR{type},
                    \BLOCK{ endfor }
                    %affinities
                    \BLOCK{ if card.primary != None }
                    primary={\VAR{card.primary}},
                    \BLOCK{ endif }
                    \BLOCK{ if card.secondary != None }
                    secondary={\VAR{card.secondary}},
                    \BLOCK{ endif }
                    \BLOCK{ if card.linked != None }
                    linked={\VAR{card.linked}},
                    \BLOCK{ endif }
                    \BLOCK{ if card.linked_type != None }
                    linked type={\VAR{card.linked_type}},
                    \BLOCK{ endif }
                    \BLOCK{ if card.linked_short == true }
                    linked short,
                    \BLOCK{ endif }
                    text={%
                        \VAR{card.text | indent(4*4)}%
                    },
                    \BLOCK{ if card.purchase != None }
                    purchase={\VAR{card.purchase}},
                    \BLOCK{ endif }
                    \BLOCK{ if card.upgrade_cost != None }
                    upgrade cost={\VAR{card.upgrade_cost}},
                    \BLOCK{ endif }
                    \BLOCK{ if card.upgrade != None }
                    upgrade={%
                        \VAR{card.upgrade | indent(4*4)}%
                    },
                    \BLOCK{ endif }
                    \BLOCK{ if card.big_art == true }
                    big art,
                    \BLOCK{ endif }
                    \BLOCK{ if card.designer != None }
                    designer={\VAR{card.designer}},
                    \BLOCK{ endif }
                ]
            \BLOCK{ endif }
        \BLOCK{ endfor }
    \BLOCK{ endif }


\BLOCK{ endfor }

\end{document} 
