\newcommand\substring[3]{%
    \directlua{ tex.sprint(string.sub ( \luastring{#1}, #2, #3 ) ) }}

\pgfkeys{/pathrow/.cd,% set the initial path
    name/.initial={},
    resources/.initial={},
    linked/.code={\pgfkeyssetvalue{linked}{#1}},
    purchase/.code={\pgfkeyssetvalue{purchase}{#1}},
    upgrade/.code={\pgfkeyssetvalue{upgrade}{#1}},
}

\newcommand{\renderthingy}[4]{%
    \begingroup
    \newcommand{\purchaseunderlay}{%
        \begin{tcbclipinterior}
            \node[
                fit={(frame.north west) (frame.south east)},
                anchor=center,
                rectangle,
                rounded corners=1mm,
                inner sep=0pt,
                % this is some weird magic to get the border to be inside the shape
                % this does mean that the line width is being halved
                % https://tex.stackexchange.com/questions/245141/pgf-tikz-draw-the-border-of-a-shape-inside-it
                preaction={clip,postaction={fill=dark!25!light!75, draw=dark, line width=0.5mm}},
                % fill=dark!25!light!75,
                % draw=dark,
                % line width=0.25mm,
            ] at (frame.center) {};
        \end{tcbclipinterior}
    }
    \bettertextbox
        [%
            y={#3}, x={#2},
            width=6mm, height=10mm, margin=1mm, 
            anchor=center,
            fontsize=14pt, 
            font=\bfseries,
        ]
        [%
            halign=center, valign=center,
            underlay={\purchaseunderlay},
            underlay={
                \begin{tcbclipframe}
                    \node[
                        anchor=north,
                        inner sep=0pt,
                        yshift=-1mm,
                        % opacity=0.5,
                    ] at (frame.north) {\includegraphics[height=3.5mm]{#4}};
                \end{tcbclipframe}
            },
            top=5mm,
        ]
    {%
        #1%
    }
    \endgroup
}

\newcommand{\renderbackgroundbox}[3]{%
    \begingroup
    \begin{tikzpicture}[x=1mm, y=1mm, overlay, remember picture]
    
        % \tikz[overlay, remember picture] 
        \node[
            anchor=west,
            rectangle,
            xshift={#1},
            yshift=-{#2},
            minimum width={#3}, 
            minimum height=6mm,
            inner sep=0mm,
            % fill=gray, 
            line width=0mm,
        ] (bb) at (current page.north west) {};


        % \tikz[overlay, remember picture] 
        \path[%
            draw=dark, line width=0.25mm, 
            fill=light,
        ] 
            ([xshift=-0.125mm, yshift=-1mm] bb.north east)
            --
            ([xshift=-1mm, yshift=-0.125mm] bb.north east)
            --
            ([xshift=1mm, yshift=-0.125mm] bb.north west)
            --
            ([xshift=0.125mm, yshift=-1mm] bb.north west)
            -- 
            ([xshift=0.125mm, yshift=1mm] bb.south west)
            --
            ([xshift=1mm, yshift=0.125mm] bb.south west)
            -- 
            ([xshift=-1mm, yshift=0.125mm] bb.south east)
            --
            ([xshift=-0.125mm, yshift=1mm] bb.south east)
            -- 
            cycle;

        % \pgfkeysifdefined{linked}{%
        %     \node[
        %         anchor=center,
        %         rectangle,
        %         xshift=0mm,
        %         yshift=-{#2-5mm},
        %         minimum width=36mm, 
        %         minimum height=3mm,
        %         inner sep=0mm,
        %         % fill=gray, 
        %         line width=0mm,
        %     ] (bb) at (current page.north) {};

        %     \path[%
        %         line width=0mm, 
        %         fill=dark!90,
        %     ] 
        %         (bb.north east)
        %         --
        %         (bb.north west)
        %         -- 
        %         ([yshift=1mm] bb.south west)
        %         --
        %         ([xshift=1mm] bb.south west)
        %         -- 
        %         ([xshift=-1mm] bb.south east)
        %         --
        %         ([yshift=1mm] bb.south east)
        %         -- 
        %         cycle;
        % }{}
    \end{tikzpicture}
    \endgroup
}

\newcommand{\renderresource}[2]{%
    \directlua{
        function get_mana_filename(s)
            symbols = {}
            % symbols["A"] = "any"
            % symbols["F"] = "focus"
            % symbols["S"] = "strength"
            % symbols["W"] = "will"
            % symbols["X"] = "spirit"
            symbols["S"] = "spirit"
            symbols["X"] = "sacrifice"
            symbols["H"] = "spirit-or-sacrifice"
            return symbols[s] or s
        end
    }

    \tikz[overlay, remember picture] 
    \node[
        anchor=north, 
        inner ysep=0pt,
        inner xsep=0pt,
        outer ysep=0pt,
        outer xsep=0pt,
        xshift={#1},
        yshift=-16mm,
    ] at (current page.north) {
        % \includesvg[width = 63mm]{frames/frame-berserker.svg}
        \includegraphics[width=4mm]{icons/mana/\directlua{tex.print(get_mana_filename(\luastring{#2}))}.png}
    };
}

\newlength{\yoffset}

\NewDocumentCommand{\pathrow}{ +O{} }{%
    % to make sure no arguments "leak" between different \card calls
    \begingroup%
    % handle key-value arguments
    \pgfkeys{/pathrow/.cd, #1}%
    %
    \setlength{\templen}{75.5mm-\yoffset}%
    % background box
    \pgfkeysifdefined{purchase}{%
        \renderbackgroundbox{9.5mm}{\templen}{\pgfkeysifdefined{upgrade}{44mm}{48mm}}
    }{%
        % \renderbackgroundbox{11.5mm}{\templen}{\pgfkeysifdefined{upgrade}{44mm}{46mm}}
        \renderbackgroundbox{8.5mm}{\templen}{\pgfkeysifdefined{upgrade}{44mm}{46mm}}
    }
    %
    % % purchase cost?
    % \pgfkeysifdefined{purchase}{%
    %     % \renderthingy{\pgfkeysvalueof{purchase}}{-23mm}{\templen}{icons/padlock.png}
    %     \renderthingy{\pgfkeysvalueof{purchase}}{-23mm}{\templen}{icons/level.png}
    % }{%
    %     \bettertextbox[%
    %         anchor=center,
    %         x=-23mm, y={\templen},
    %         width=7mm, height=2mm, margin=0mm,
    %         fontsize=7pt, font=\titlefont\sbseries,
    %         rotate=90,
    %     ]
    %     [halign=center, valign=centerline]
    %     {Starter}%
    % }%
    % %
    % % upgrade cost?
    % \pgfkeysifdefined{upgrade}{%
    %     % \renderthingy{\pgfkeysvalueof{upgrade}}{23mm}{\templen}{icons/upgrade.png}
    %     \renderthingy{\pgfkeysvalueof{upgrade}}{23mm}{\templen}{icons/level.png}
    % }{}%
    %
    % card name
    \bettertextbox[%
        anchor=center,
        y={\templen}, 
        width=44mm, height=7mm, margin=1mm, 
        fontsize=10pt, font=\titlefont,
    ]
    [valign=centerline]
    {%
        \pgfkeysvalueof{/pathrow/name}%
    }%
    %
    % linked cards
    % \pgfkeysifdefined{linked}{%
    %     \bettertextbox[%
    %         anchor=center,
    %         y={\templen-5mm}, 
    %         width=36mm, height=3mm, margin=0.25mm, 
    %         fontsize=7pt, font=\titlefont\sbseries,
    %     ]
    %     [valign=centerline, left=0.75mm, right=0.75mm]
    %     {%
    %         \textcolor{light}{\pgfkeysvalueof{linked}}%
    %     }%
    % }{}%
    \endgroup%
}


\pgfkeys{/path/.cd,% set the initial path
    name/.initial={},
    resources/.initial={},
    color-left/.initial=FF00FF,
    color-right/.initial=0000FF,
    % purchase/.code={\pgfkeyssetvalue{purchase}{#1}},
    % upgrade/.code={\pgfkeyssetvalue{upgrade}{#1}},
    passive name/.code={\pgfkeyssetvalue{passive name}{#1}},
    passive/.code={\pgfkeyssetvalue{passive}{#1}},
    big passive/.code={\pgfkeyssetvalue{big passive}{#1}},%
    % affinities
    primary/.code={\pgfkeyssetvalue{primary}{#1}},
    secondary/.code={\pgfkeyssetvalue{secondary}{#1}},
}

\NewDocumentCommand{\pathcard}{ +O{} +m }{

    % start a new page if we have to
    \clearpage

    % to make sure no arguments "leak" between different \card calls
    \begingroup
    % handle key-value arguments
    \pgfkeys{/path/.cd, #1}
    
    % increase our page counter and track card
    \countpage
    \registercard{\pgfkeysvalueof{/path/name}}{\pgfkeysvalueof{/path/name}}

    \colorlet{dark}{black}
    \colorlet{light}{white}
    \colorlet{sequencecolor}{dark!70}
    \colorlet{ugradegray}{dark!25!light!75}

    % render frame
    % the tikz stuff makes sure it's rendered in the background and not
    % as part of the document flow
    
    \colorlet{primary}{\pgfkeysifdefined{primary}{\pgfkeysvalueof{primary}}{noaffinity}}
    \colorlet{secondary}{\pgfkeysifdefined{secondary}{\pgfkeysvalueof{secondary}}{\pgfkeysifdefined{primary}{\pgfkeysvalueof{primary}}{noaffinity}}}

    \getColorSpec[\mycolorprimary]{HTML}{primary}
    \getColorSpec[\mycolorsecondary]{HTML}{secondary}

    \pgfkeys{/path/.cd,
        color-left=\mycolorprimary,
        color-right=\mycolorsecondary,
    }

    \pgfkeys{/card/.cd,
        color-left=\mycolorprimary,
        color-right=\mycolorsecondary,
    }

    \global\let\domainprimary\mycolorprimary
    \global\let\domainsecondary\mycolorsecondary

    \colorlet{primarylight}{primary!60!white}
    \colorlet{secondarylight}{secondary!60!white}

    \colorlet{left}{primary}
    \colorlet{right}{\pgfkeysifdefined{secondary}{secondarylight}{primarylight}}


    \tikz[overlay, remember picture] 
    \node[
        anchor=north,
        yshift=-1.5mm,
        rectangle, 
        minimum width=\paperwidth-3mm, 
        minimum height=\paperheight-3mm,
        left color=left, 
        right color=right,
        shading = axis,
        shading angle=135, 
    ] at (current page.north) {};
    
    % frame (with transparency):
    \tikz[overlay, remember picture] 
    \node[
        anchor=north, 
        inner ysep=0pt,
        inner xsep=0pt,
        outer ysep=0pt,
        outer xsep=0pt,
    ] at (current page.north) {
        % \includegraphics[width=63mm]{icons/frame2.png}
        \includegraphics[width=63mm]{%
            % icons/frame2-path.png
            icons/frame2-path-title-wide.png
        }
    };

    % resources
    % \typeout{\pgfkeysvalueof{/path/resources}}
    % \renderresource{-5mm}{\substring{\pgfkeysvalueof{/path/resources}}{1}{1}}
    % \renderresource{0mm}{\substring{\pgfkeysvalueof{/path/resources}}{2}{2}}
    % \renderresource{5mm}{\substring{\pgfkeysvalueof{/path/resources}}{3}{3}}


    % \renderaffinities

    \pgfkeysifdefined{primary}{%
        \begin{tikzpicture}[x=1mm, y=1mm, overlay, remember picture]
            % \node[
            %     anchor=center,
            %     rectangle,
            %     rounded corners=1mm,
            %     xshift=-5.5mm-3mm,
            %     yshift=-5.5mm-4mm,
            %     minimum width=6mm, 
            %     minimum height=8mm,
            %     inner sep=0mm,
            %     % fill=dark!25!light!75,
            %     fill={\pgfkeysifdefined{secondary}{secondarylight}{primarylight}},
            %     line width=0mm,
            % ] (bb) at (current page.north east) {};

            \node[
                opacity=0.07,
                anchor=center,
                inner sep=0pt,
                yshift=-36mm,
            ] at (current page.north) {
                \includegraphics[width=36mm, height=36mm, keepaspectratio]{icons/affinities/\pgfkeysvalueof{\pgfkeysifdefined{secondary}{secondary}{primary}}}
            };
        \end{tikzpicture}
    }{}
    
    % make sure we have the right font by default
    \bodyfont
    
    % % name
    % \bettertextbox
    %     [y=8mm, width=36mm, height=7mm, margin=0mm, fontsize=18pt, font=\titlefont\bfseries]
    %     [halign=center, valign=centerline]
    % {%
    %     \pgfkeysvalueof{/path/name}
    % }
     
    % % "path of the"
    % \bettertextbox
    %     [y=5mm, width=36mm, height=4mm, margin=0mm, fontsize=8pt, font=\titlefont\sbseries]
    %     [halign=center, valign=centerline]
    % {Domain of the}

    % name
    \bettertextbox
        [y=4.5mm, width=42mm, height=6mm, margin=0mm, fontsize=14pt, font=\titlefont\bfseries]
        [halign=center, valign=centerline]
    {%
        \pgfkeysvalueof{/path/name}%
    }
     
    % type
    \bettertextbox
        [y=11mm, width=36mm, height=4mm, margin=0mm, fontsize=8pt, font=\titlefont\sbseries\itshape]
        [halign=center, valign=centerline]
    {%
        Domain%
    }

    \pgfkeysifdefined{passive}{%
        % \bettertextbox[%
        %     y=20mm,
        %     width=40mm, height=4mm, margin=0mm, 
        %     fontsize=10pt,
        %     font=\titlefont\sbseries,
        % ]
        % [halign=center, valign=centerline]{%
        %     \pgfkeysvalueof{passive name}:%
        % }
        
        \bettertextbox[%
            anchor=center,
            y=\pgfkeysifdefined{big passive}{40mm}{36mm},
            width=44mm, height=\pgfkeysifdefined{big passive}{44mm}{36mm}, margin=1mm, 
            fontsize=10pt,
        ]
        [valign=center]{%
            \pgfkeysvalueof{passive}
        }
    }{}%

    \setlength{\yoffset}{0mm}

    % render rows
    #2%

    % path
    \bettertextbox
        [y=81.5mm, x=-27mm, width=24mm, height=4mm, anchor=north west, font=\titlefont\color{white}\sbseries, fontsize=7pt, margin=0.5mm]
        [valign=centerline, left=-0.5mm]
    {%
        \pgfkeysvalueof{/path/name}
    }
    
    % date - glorybound
    \bettertextbox
        [y=81.5mm, x=27mm, width=24mm, height=4mm, anchor=north east, font=\titlefont\color{white}\sbseries, fontsize=7pt, margin=0.5mm]
        [valign=centerline, right=-0.5mm, halign=flush right]
    {%
        \today\ \ --\ \ GLORYBOUND
        % \today\ \ --\ \ Glorybound
    }

    \endgroup
}
