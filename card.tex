%%% Commands to be used in card definitions

% \newcommand{\block}{{\par\centering\textbf{Block} an attack.\par}}

% \newcommand{\attack}[1]{{\par\centering\textbf{Attack} #1.\par}}

\newcommand{\block}{%
    \begin{tikzpicture}[x=1mm, y=1mm]
        \node[
            anchor=center,
            rectangle,
            minimum width=\linewidth, 
            minimum height=6mm,
            % fill=gray, 
        ] (bb) {};

        \path[
            draw=black, line width=0.5mm, 
            fill=black
        ] 
            ([xshift=-1.5mm, yshift=-0.25mm] bb.north east) coordinate (start)
            --
            ([xshift=1.5mm, yshift=-0.25mm] bb.north west)
            --
            ([xshift=0.25mm, yshift=-1.5mm] bb.north west)
            -- 
            ([xshift=0.25mm, yshift=1.5mm] bb.south west)
            --
            ([xshift=1.5mm, yshift=0.25mm] bb.south west)
            -- 
            ([xshift=-1.5mm, yshift=0.25mm] bb.south east)
            -- 
            (start);

        \node[
            anchor=west,
            minimum width=\linewidth-8mm,
        ] at (bb.west) {
            \fontsize{10pt}{10pt}\selectfont
            \textcolor{white}{\textbf{Block} an attack.}
        };

        % block icon
        \node[
            anchor=east, 
            inner ysep=0pt,
            inner xsep=0pt,
            outer ysep=0pt,
            outer xsep=0pt,
        ] (icon) at (bb.east) {
            \includegraphics[width=10mm]{\geticon{block}{\pgfkeysvalueof{/card/color-left}}{\pgfkeysvalueof{/card/color-right}}}
        };

        % and the white-ish semi-transparent overlay to make the colors a bit more wahsed out
        \node[
            anchor=east, 
            inner ysep=0pt,
            inner xsep=0pt,
            outer ysep=0pt,
            outer xsep=0pt,
            opacity = 0.5,
        ] at (bb.east) {
            \includegraphics[width=10mm]{\geticon{block}{FFFFFF}{FFFFFF}}
        };
    \end{tikzpicture}%
}

\newcommand{\attack}[1]{%
    \begin{tikzpicture}[x=1mm, y=1mm]
        \node[
            anchor=center,
            rectangle,
            minimum width=\linewidth, 
            minimum height=6mm,
            % fill=gray, 
        ] (bb) {};

        \path[
            draw=black, line width=0.5mm, 
            % fill=black
        ] 
            ([xshift=-1.5mm, yshift=-0.25mm] bb.north east) coordinate (start)
            --
            ([xshift=1.5mm, yshift=-0.25mm] bb.north west)
            --
            ([xshift=0.25mm, yshift=-1.5mm] bb.north west)
            -- 
            ([xshift=0.25mm, yshift=1.5mm] bb.south west)
            --
            ([xshift=1.5mm, yshift=0.25mm] bb.south west)
            -- 
            ([xshift=-1.5mm, yshift=0.25mm] bb.south east)
            -- 
            (start);

        \node[
            anchor=west,
            minimum width=\linewidth-8mm,
        ] at (bb.west) {
            \fontsize{10pt}{10pt}\selectfont
            \textcolor{black}{\textbf{Attack}}
        };

        % attack icon
        \node[
            anchor=east, 
            inner ysep=0pt,
            inner xsep=0pt,
            outer ysep=0pt,
            outer xsep=0pt,
        ] (icon) at (bb.east) {
            \includegraphics[width=10mm]{\geticon{attack}{\pgfkeysvalueof{/card/color-left}}{\pgfkeysvalueof{/card/color-right}}}
        };

        % and the white-ish semi-transparent overlay to make the colors a bit more wahsed out
        \node[
            anchor=east, 
            inner ysep=0pt,
            inner xsep=0pt,
            outer ysep=0pt,
            outer xsep=0pt,
            opacity = 0.5,
        ] at (bb.east) {
            \includegraphics[width=10mm]{\geticon{attack}{FFFFFF}{FFFFFF}}
        };

        % https://tex.stackexchange.com/questions/400296/outline-text-characters
        \node[
            anchor=center,
            minimum width=10mm, 
            minimum height=10mm,
        ] at (icon.center) {%
            \fontsize{24pt}{24pt}\selectfont
            \bfseries
            \textpdfrender{
                TextRenderingMode=FillStrokeClip,
                LineWidth=0.5mm,
                FillColor=white,
                StrokeColor=black, 
                MiterLimit=1,
            }{#1}
        };
        \node[
            anchor=center,
            minimum width=10mm, 
            minimum height=10mm,
        ] at (icon.center) {%
            \fontsize{24pt}{24pt}\selectfont
            \bfseries
            \textcolor{white}{#1}
        };
    \end{tikzpicture}%
}

\renewcommand\tabularxcolumn[1]{m{#1}}% for vertical centering text in X column
\renewcommand{\arraystretch}{1.5}
\NewDocumentCommand{\sequence}{+o +o +o}{%
    \begin{tabularx}{\textwidth}{ m{3mm}>{\raggedright\arraybackslash}X }
        [1] & #1 \\
        \IfValueT{#2}
        {%
            [2] & #2 \\
            \IfValueT{#3}
            {%
                [3] & #3 \\
            }
        }
    \end{tabularx}
}

%%%

\newcommand{\setpath}[3]{
    \pgfkeys{/card/.cd,
        path=#1,
        color-left=#2,
        color-right=#3,
    }
}

% \begin{luacode*}
    
% \end{luacode*}

\newcommand{\rendermanasymbol}[2]{
    \directlua{
        function get_mana_filename(s)
            symbols = {}
            symbols["A"] = "any"
            symbols["F"] = "focus"
            symbols["S"] = "strength"
            symbols["W"] = "will"
            % symbols["S"] = "spirit"
            return symbols[s] or s
        end
    }
    \tikz[overlay, remember picture] 
    \node[
        anchor=north west, 
        inner ysep=0pt,
        inner xsep=0pt,
        outer ysep=0pt,
        outer xsep=0pt,
        xshift=5.5mm,
        yshift=-\directlua{tex.print(5.5 + 6 * (#1 - 1))}mm,
    ] at (current page.north west) {
        % \includesvg[width = 63mm]{frames/frame-berserker.svg}
        \includegraphics[width=5mm]{icons/mana/\directlua{tex.print(get_mana_filename(\luastringN{#2}))}.png}
    };
}

\begin{luacode*}
    function render_mana(s)
        for i = 1, #s do
            tex.print("\\rendermanasymbol{"..i.."}{"..s:sub(i,i).."}")
        end
    end
\end{luacode*}
\newcommand{\rendermana}[1]{\directlua{render_mana(\luastring{#1})}}


\pgfkeys{/card/.cd,% set the initial path
    name/.initial={},
    % type/.initial={},
    cost/.initial={},
    text/.initial={},
    path/.initial={},
    color-left/.initial=FF00FF,
    color-right/.initial=0000FF,
    permanent/.code={\pgfkeyssetvalue{permanent}{#1}},
    sequence/.code={\pgfkeyssetvalue{sequence}{#1}},
    oneshot/.code={\pgfkeyssetvalue{oneshot}{#1}},
    innate/.code={\pgfkeyssetvalue{innate}{#1}},
    extra/.code={\pgfkeyssetvalue{extra}{#1}},
    % read-only key to get the printed type line
    type/.initial={%
        % supertypes
        \pgfkeysifdefined{innate}{Innate }{}%
        \pgfkeysifdefined{extra}{Extra }{}%
        % base types
        \pgfkeysifdefined{permanent}{Permanent }{}%
        \pgfkeysifdefined{sequence}{Sequence }{}%
    }
}

% building the attack and block icons
% generate gradient png
% magick -size 474x473 -define gradient:direction=Northeast gradient:"#00DB1D"-"#BF00FF" gradient.png

% two commands for the overlaying
% convert gradient.png attack.png -composite resulta.png
% convert resulta.png attack-mask-2.png  -compose copy_opacity -composite result.png

% and this just combines both of them into one
% convert gradient.png attack.png -composite attack-mask.png -compose copy_opacity -composite result.png

% and then this combines all three into one:
% convert -size 474x473 -define gradient:direction=Northeast gradient:"#00DB1D"-"#BF00FF" attack.png -composite attack-mask.png -compose copy_opacity -composite result.png


\begin{luacode*}
    function generate_icon(name, col1, col2)
        io.popen("cd icons/ && convert -size 473x473 -define gradient:direction=Northeast gradient:\"#"..col1.."\"-\"#"..col2.."\" "..name..".png -composite "..name.."-mask.png -compose copy_opacity -composite "..name.."-"..col1.."-"..col2..".png")
    end
    function get_icon(name, col1, col2)
        filename = "icons/"..name.."-"..col1.."-"..col2..".png"
        -- check if it already exists
        local f = io.open(filename, "r")
        if f~=nil then 
            io.close(f) 
            return filename
        else 
            generate_icon(name, col1, col2)
            return filename  
        end
    end
\end{luacode*}
\newcommand{\geticon}[3]{\directlua{tex.print(get_icon(\luastring{#1}, \luastring{#2}, \luastring{#3}))}}

\NewDocumentCommand{\card}{ +O{} }{

    % start a new page if we have to
    \clearpage

    % to make sure no arguments "leak" between different \card calls
    \begingroup
    % handle key-value arguments
    \pgfkeys{/card/.cd, #1}
    
    % render frame
    % the tikz stuff makes sure it's rendered in the background and not
    % as part of the document flow
    
    % background gradient:
    \definecolor{left}  {HTML}{\pgfkeysvalueof{/card/color-left}}
    \definecolor{right} {HTML}{\pgfkeysvalueof{/card/color-right}}
    \tikz[overlay, remember picture] 
    \node[
        anchor=north,
        yshift=-1.5mm,
        rectangle, 
        minimum width=\paperwidth-3mm, 
        minimum height=\paperheight-3mm,
        left color=left, 
        right color=right,
        shading = axis,
        shading angle=135, 
    ] at (current page.north) {};
    
    % frame (with transparency):
    \tikz[overlay, remember picture] 
    \node[
        anchor=north, 
        inner ysep=0pt,
        inner xsep=0pt,
        outer ysep=0pt,
        outer xsep=0pt,
    ] at (current page.north) {
        % \includesvg[width = 63mm]{frames/frame-berserker.svg}
        \includegraphics[width=63mm]{frame2.png}
    };

    % art:
    % (just a placeholder rectangle currently)
    \tikz[overlay, remember picture] 
    \node[
        anchor=north,
        yshift=-17mm,
        rectangle, 
        minimum width=36mm, 
        minimum height=30mm,
        left color=left, 
        right color=right,
        shading = axis,
        shading angle=135, 
        opacity = 0.3,
    ] at (current page.north) {};


    % cost
    \rendermana{\pgfkeysvalueof{/card/cost}}
    % \rendermanasymbol{1}{W}
    % \rendermanasymbol{2}{S}
    % \rendermanasymbol{3}{A}
    
    % make sure we have the right font ny default
    \bodyfont
    
    % name
    \bettertextbox
        [y=4.5mm, width=36mm, height=6mm, margin=0mm, fontsize=14pt, font=\titlefont\bfseries]
        [halign=center, valign=centerline]
    {%
        \pgfkeysvalueof{/card/name}
    }
     
    % type
    \bettertextbox
        [y=11mm, width=36mm, height=4mm, margin=0mm, fontsize=8pt, font=\titlefont\sbseries\itshape]
        [halign=center, valign=centerline]
    {%
        \pgfkeysvalueof{/card/type}
    }

    % oneshot?
    \pgfkeysifdefined{oneshot}{%
        % defined 
        \tikz[overlay, remember picture] 
        \node[
            anchor=south, 
            inner ysep=0pt,
            inner xsep=0pt,
            outer ysep=0pt,
            outer xsep=0pt,
            yshift=7.5mm,
        ] at (current page.south) {
            \includegraphics[width=36mm]{oneshot-banner.png}
        };
    }{%
        % not defined
    }

    % text

    \bettertextbox
        [%
            y=\pgfkeysifdefined{oneshot}{76.5mm}{81.5mm}, 
            width=54mm, 
            % height=\pgfkeysifdefined{oneshot}{24mm}{32mm}, 
            height=\pgfkeysifdefined{oneshot}{28mm}{34mm}, 
            anchor=south, 
            margin=2mm,
        ]
        [valign=center, bottom=0.25mm, % the bottom=... just adds some extra spacing there
            % top=-2mm,
            underlay=\pgfkeysifdefined{permanent}{%
                \begin{tcbclipinterior}
                    \node[
                        anchor=center, 
                        opacity=0.15,
                        % yshift=1mm
                    ] at (frame.center) {
                        \includegraphics[height=14mm]{permanent-watermark.png}
                    };
                \end{tcbclipinterior}
            }{}
        ] 
    {%
        \pgfkeysvalueof{/card/text}
    }
    
    % path
    \bettertextbox
        [y=81.5mm, x=-27mm, width=24mm, height=4mm, anchor=north west, font=\titlefont\color{white}\sbseries, fontsize=7pt, margin=0.5mm]
        [valign=centerline, left=-0.5mm]
    {%
        \pgfkeysvalueof{/card/path}
    }
    
    % date - glorybound
    \bettertextbox
        [y=81.5mm, x=27mm, width=24mm, height=4mm, anchor=north east, font=\titlefont\color{white}\sbseries, fontsize=7pt, margin=0.5mm]
        [valign=centerline, right=-0.5mm, halign=flush right]
    {%
        \today\ \ --\ \ Glorybound
    }

    \endgroup
}
