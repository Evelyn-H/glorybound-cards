% !TeX program = lualatex

\documentclass{article}
% \documentclass{minimal}
\usepackage[
    paperwidth=63mm, 
    paperheight=88mm,
    margin=0mm,
]{geometry}
\setlength\parindent{0pt}
\setlength{\parskip}{2mm}

% the cmyk option makes it convert rgb color to cmyk space
% this makes the pdf output look more "realistic", but less vibrant
% \usepackage[cmyk]{xcolor}
\usepackage{xcolor}

% makes lua stuff easier
\usepackage{luacode}

% for \NewDocumentCommand
\usepackage{xparse}

% for filler text
\usepackage{lipsum}

% for getting the current date
\usepackage[datesep=/]{datetime2}
\DTMsetdatestyle{mmddyy}

% for textboxes
% this also pulls in tikz as a dependency
\usepackage[fitting, skins]{tcolorbox}
\usetikzlibrary{backgrounds}
\usetikzlibrary{fit}

% for length maths shenanigans
\usepackage{calc}

% for tables stuff
\usepackage{tabularx}
\usepackage{array}

% for outlined text
\usepackage{pdfrender}

% for scaling things
\usepackage{scalerel}

% this gets rid of a bunch of font scaling warnings 
% by replacing the standard fonts with scalable versions
\usepackage{lmodern}

\usepackage{ifthen}

% load fonts
\usepackage{fontspec}
\newfontfamily\bodyfont{CrimsonPro-Regular}[
    Path           = "fonts/Crimson_Pro/",
    Extension      = .ttf,
    Ligatures      = TeX,
    Numbers        = Monospaced,
    BoldFont       = CrimsonPro-Bold,
    ItalicFont     = CrimsonPro-Italic,
    BoldItalicFont = CrimsonPro-BoldItalic,
    FontFace       = {sb}{n}{CrimsonPro-SemiBold},
    FontFace       = {sb}{it}{CrimsonPro-SemiBoldItalic},
]
\newfontfamily\titlefont{Grenze-Regular}[
    Path           = "fonts/Grenze/",
    Extension      = .ttf,
    Ligatures      = TeX,
    BoldFont       = Grenze-Bold,
    ItalicFont     = Grenze-Italic,
    BoldItalicFont = Grenze-BoldItalic,
    FontFace       = {sb}{n}{Grenze-SemiBold},
    FontFace       = {sb}{it}{Grenze-SemiBoldItalic},
]

% semibold commands
\DeclareOldFontCommand{\sbseries}{\fontseries{sb}\selectfont}{\mathbf}
\DeclareTextFontCommand{\textsb}{\sbseries}


% my custom textbox command
% Some tcolorbox hackery to add "proper" one-line vertical centering
\newlength{\Eheight}
\makeatletter
\def\tcb@dbox@top#1#2#3#4#5{\pgftext[x=#1,y=#2+#3,left,top]{#5}}
% define new valign type
\def\tcb@dbox@centerline#1#2#3#4#5{%
    % based on valign=top 
    \pgftext[x=#1,y=#2+#3,left,top]{%
        % make sure we have the correct front set for measuring 
        \pgfkeysvalueof{/textbox/font}%
        \fontsize{\tcbfitdim}{\tcbfitdim}\selectfont%
        % measure cap height
        \setlength{\Eheight}{\heightof{E}}%
        % \the\Eheight
        % and then raise (actually lower) the text based on that
        \raisebox{%
            0pt-0.5\Eheight-(\pgfkeysvalueof{/textbox/height}-\pgfkeysvalueof{/textbox/margin}-\pgfkeysvalueof{/textbox/margin})/2
        }{#5}
    }
}
\def\tcb@dbox@center#1#2#3#4#5{\pgftext[x=#1,y=#2+#3/2,left]{#5}}
\def\tcb@dbox@bottom#1#2#3#4#5{\pgftext[x=#1,y=#2,left,bottom]{#5}}

\tcbset{
    % add new valign type to the list of valid options
    valign/centerline/.code={\def\kvtcb@valignupper{centerline}}
}
\makeatother


\pgfkeys{/textbox/.cd,% set the initial path
    anchor/.initial=north,
    anchorpos/.initial=north,
    x/.initial=0mm,
    y/.initial=0mm,
    rotate/.initial=0,
    width/.initial=54mm,
    height/.initial=4mm,
    margin/.initial=1mm,
    fontsize/.initial=10pt,
    font/.initial=\bodyfont,
}

% first argument (optional): key-value options
% second argument (optional): extra stuff passed to tcolorbox
%third argument: textbox contents
\NewDocumentCommand{\bettertextbox}{ O{} O{} +m }{
    \begingroup%
    % handle key-value arguments
    \pgfkeys{/textbox/.cd, #1}%
% 
    \tikz[overlay, remember picture] 
    \node[
        anchor=\pgfkeysvalueof{/textbox/anchor}, 
        xshift=\pgfkeysvalueof{/textbox/x},
        yshift= -1 * \pgfkeysvalueof{/textbox/y},
        inner ysep=0pt,
        inner xsep=0pt,
        outer ysep=0pt,
        outer xsep=0pt,
        rotate=\pgfkeysvalueof{/textbox/rotate},
    ] at (current page.\pgfkeysvalueof{/textbox/anchorpos}) {%
        \begin{tcolorbox}[
            % show bounding box=green,
            width=\pgfkeysvalueof{/textbox/width},
            height=\pgfkeysvalueof{/textbox/height},
            fit,
            fit basedim=\pgfkeysvalueof{/textbox/fontsize},
            fit skip=1.1, % baselineskip
            blank,
            parbox=false, % makes parskip work
            halign=flush left,
            boxsep=\pgfkeysvalueof{/textbox/margin},
            % fontupper=\fontsize{\pgfkeysvalueof{/textbox/fontsize}}{\pgfkeysvalueof{/textbox/fontsize}}\selectfont,
            % tikz={opacity=0.5,transparency group},
            #2
        ]%%
            \color{dark}
            \pgfkeysvalueof{/textbox/font}%
            % this lets the parskip adapt to scaling text, 
            % it makes the whitespace look more natural 
            % and leaves more space for text
            \setlength{\parskip}{0.5\tcbfitdim}%
            #3
        \end{tcolorbox}%
    };%
    \endgroup%
}


% helpers for file splitting / output shenanigans
\begin{luacode*}
    start_page = 0
    end_page = 0
    file_name = nil
    -- the "w" open mode erases old content
    page_file = io.open("page-numbers-paths.log", "w") 
    card_file = io.open("page-numbers-cards.log", "w") 
    current_page = 0
\end{luacode*}


\newcommand{\countpage}{%
    \directlua{
        current_page = current_page + 1
    }
}

\newcommand{\registercard}[2]{%
    \directlua{
        path_name = \luastring{#1}
        card_name = \luastring{#2}
        card_file:write(path_name..' - '..card_name..' - '..current_page..'\string\n')
    }
}

% \newcommand{\beginpathfile}[1]{%
%     \directlua{
%         file_name = \luastring{#1}
%         start_page = current_page + 1
%     }
% }

% \newcommand{\finishpathfile}{%
%     \directlua{
%         end_page = current_page
%         if file_name then
%             page_file:write(file_name..' '..start_page..' '..end_page..'\string\n')
%         end
%     }
% }


% card rendering code
%%% Commands to be used in card definitions

% \newcommand{\block}{{\par\centering\textbf{Block} an attack.\par}}

% \newcommand{\attack}[1]{{\par\centering\textbf{Attack} #1.\par}}

\pgfkeys{/fancyrenderer/.cd,% set the initial path
    text/.initial={},
    icon/.initial={},
    icontext/.code={\pgfkeyssetvalue{/fancyrenderer/icontext}{#1}},
    fillcolor/.initial={white},
    textcolor/.initial={black},
    scale/.initial={0.8},
    % scale/.initial={0.8},
}

% helper lengths
\newlength\cornersize
\newlength\iconwidth
\newlength\tempfontsize
\newlength\outlinewidth

% https://tex.stackexchange.com/a/6424
\makeatletter
\newcommand*{\DivideLengths}[2]{%
  \strip@pt\dimexpr\number\numexpr\number\dimexpr#1\relax*65536/\number\dimexpr#2\relax\relax sp\relax
}
\makeatother

\NewDocumentCommand{\fancyrenderer}{ +O{} }{%
    % some magic for autoscaling
    \pgfkeyssetvalue{/fancyrenderer/scale}{%
        \directlua{%
            ratio = \DivideLengths{\tcbfitdim}{\pgfkeysvalueof{/textbox/fontsize}}
            % tex.print(ratio)
            if ratio > 0.9 then
                tex.print("1.0")
            % elseif ratio < 0.7 then
            %     tex.print("0.6")
            else
                tex.print("0.8")
            end
        }
    }
    % \pgfkeyssetvalue{/fancyrenderer/scale}{\DivideLengths{\tcbfitdim}{\pgfkeysvalueof{/textbox/fontsize}}}
    % horizontal centering (relative to the parent parbox)
    \centering    
    % to make sure no arguments "leak" between different calls
    \begingroup
    % handle key-value arguments
    \pgfkeys{/fancyrenderer/.cd, #1}
    \begin{tikzpicture}[x=1mm, y=1mm]
        \node[
            anchor=center,
            rectangle,
            % minimum width=0.8\linewidth, 
            minimum width=36mm, 
            minimum height=6mm*\pgfkeysvalueof{/fancyrenderer/scale},
            % fill=gray, 
        ] (bb) {};


        \setlength\cornersize{1.5mm*\real{\pgfkeysvalueof{/fancyrenderer/scale}}}
        \path[
            draw=black, line width=0.5mm, 
            fill=\pgfkeysvalueof{/fancyrenderer/fillcolor}
        ] 
            ([xshift=-\cornersize, yshift=-0.25mm] bb.north east) coordinate (start)
            --
            ([xshift=\cornersize, yshift=-0.25mm] bb.north west)
            --
            ([xshift=0.25mm, yshift=-\cornersize] bb.north west)
            -- 
            ([xshift=0.25mm, yshift=\cornersize] bb.south west)
            --
            ([xshift=\cornersize, yshift=0.25mm] bb.south west)
            -- 
            ([xshift=-\cornersize, yshift=0.25mm] bb.south east)
            -- 
            (start);

        \setlength\tempfontsize{10pt*\real{\pgfkeysvalueof{/fancyrenderer/scale}}}

        \node[
            anchor=west,
            % anchor=mid west,
            % minimum width=0.8\linewidth-8mm,
            minimum width=36mm-8mm,
        ] at (bb.west) {
            \fontsize{\tempfontsize}{\tempfontsize}\selectfont
            \textcolor{\pgfkeysvalueof{/fancyrenderer/textcolor}}{\pgfkeysvalueof{/fancyrenderer/text}}
        };

        \setlength\iconwidth{10mm*\real{\pgfkeysvalueof{/fancyrenderer/scale}}}

        % attack icon
        \node[
            anchor=east, 
            inner ysep=0pt,
            inner xsep=0pt,
            outer ysep=0pt,
            outer xsep=0pt,
        ] (icon) at (bb.east) {
            \includegraphics[width=\iconwidth]{\geticon{\pgfkeysvalueof{/fancyrenderer/icon}}{\pgfkeysvalueof{/card/color-left}}{\pgfkeysvalueof{/card/color-right}}}
        };

        % and the white-ish semi-transparent overlay to make the colors a bit more wahsed out
        \node[
            anchor=east, 
            inner ysep=0pt,
            inner xsep=0pt,
            outer ysep=0pt,
            outer xsep=0pt,
            opacity = 0.5,
        ] at (bb.east) {
            \includegraphics[width=\iconwidth]{\geticon{\pgfkeysvalueof{/fancyrenderer/icon}}{FFFFFF}{FFFFFF}}
        };

        \setlength\tempfontsize{24pt*\real{\pgfkeysvalueof{/fancyrenderer/scale}}}
        \setlength\outlinewidth{0.75mm*\real{\pgfkeysvalueof{/fancyrenderer/scale}}}

        \pgfkeysifdefined{/fancyrenderer/icontext}{
            % https://tex.stackexchange.com/questions/400296/outline-text-characters
            \node[
                anchor=center,
                minimum width=\iconwidth, 
                minimum height=\iconwidth,
            ] at (icon.center) {%
                \fontsize{\tempfontsize}{\tempfontsize}\selectfont
                \bfseries
                \textpdfrender{
                    TextRenderingMode=FillStrokeClip,
                    LineWidth=\outlinewidth,
                    FillColor=white,
                    StrokeColor=black, 
                    MiterLimit=1,
                }{\pgfkeysvalueof{/fancyrenderer/icontext}}
            };
            \node[
                anchor=center,
                minimum width=\iconwidth, 
                minimum height=\iconwidth,
            ] at (icon.center) {%
                \fontsize{\tempfontsize}{\tempfontsize}\selectfont
                \bfseries
                \textcolor{white}{\pgfkeysvalueof{/fancyrenderer/icontext}}
            };
        }{}
        
    \end{tikzpicture}%
    \endgroup
}

\newcommand{\block}{%
    \fancyrenderer[%
        text={\textbf{Block} an attack.},
        icon=block,
        fillcolor=black,
        textcolor=white,
    ]
}
\newcommand{\attack}[1]{%
    \fancyrenderer[%
        text={\textbf{Attack}},
        icon=attack,
        icontext={#1},
    ]
}

\renewcommand\tabularxcolumn[1]{m{#1}}% for vertical centering text in X column
\renewcommand{\arraystretch}{1.5}
\NewDocumentCommand{\sequence}{+o +o +o}{%
    \begin{tabularx}{\textwidth}{ m{3mm}>{\raggedright\arraybackslash}X }
        [1] & {#1} \\
        \IfValueT{#2}
        {%
            [2] & {#2} \\
            \IfValueT{#3}
            {%
                [3] & {#3} \\
            }
        }
    \end{tabularx}
}

%%%


%%% Misc rendering helper functions

\newcommand{\rendermanasymbol}[2]{
    \directlua{
        function get_mana_filename(s)
            symbols = {}
            symbols["A"] = "any"
            symbols["F"] = "focus"
            symbols["S"] = "strength"
            symbols["W"] = "will"
            % symbols["S"] = "spirit"
            return symbols[s] or s
        end
    }
    \tikz[overlay, remember picture] 
    \node[
        anchor=north west, 
        inner ysep=0pt,
        inner xsep=0pt,
        outer ysep=0pt,
        outer xsep=0pt,
        xshift=5.5mm,
        yshift=-\directlua{tex.print(5.5 + 6 * (#1 - 1))}mm,
    ] at (current page.north west) {
        % \includesvg[width = 63mm]{frames/frame-berserker.svg}
        \includegraphics[width=5mm]{icons/mana/\directlua{tex.print(get_mana_filename(\luastringN{#2}))}.png}
    };
}

\begin{luacode*}
    function render_mana(s)
        for i = 1, #s do
            tex.print("\\rendermanasymbol{"..i.."}{"..s:sub(i,i).."}")
        end
    end
\end{luacode*}
\newcommand{\rendermana}[1]{\directlua{render_mana(\luastring{#1})}}


% building the attack and block icons
% generate gradient png
% magick -size 474x473 -define gradient:direction=Northeast gradient:"#00DB1D"-"#BF00FF" gradient.png

% two commands for the overlaying
% convert gradient.png attack.png -composite resulta.png
% convert resulta.png attack-mask-2.png  -compose copy_opacity -composite result.png

% and this just combines both of them into one
% convert gradient.png attack.png -composite attack-mask.png -compose copy_opacity -composite result.png

% and then this combines all three into one:
% convert -size 474x473 -define gradient:direction=Northeast gradient:"#00DB1D"-"#BF00FF" attack.png -composite attack-mask.png -compose copy_opacity -composite result.png


\begin{luacode*}
    function generate_icon(name, col1, col2)
        io.popen("cd icons/ && convert -size 473x473 -define gradient:direction=Northeast gradient:\"#"..col1.."\"-\"#"..col2.."\" "..name..".png -composite "..name.."-mask.png -compose copy_opacity -composite "..name.."-"..col1.."-"..col2..".png")
    end
    function get_icon(name, col1, col2)
        filename = "icons/"..name.."-"..col1.."-"..col2..".png"
        -- check if it already exists
        local f = io.open(filename, "r")
        if f~=nil then 
            io.close(f) 
            return filename
        else 
            generate_icon(name, col1, col2)
            return filename  
        end
    end
\end{luacode*}
\newcommand{\geticon}[3]{\directlua{tex.print(get_icon(\luastring{#1}, \luastring{#2}, \luastring{#3}))}}

\pgfkeys{/card/.cd,% set the initial path
    name/.initial={},
    % type/.initial={},
    cost/.initial={},
    text/.initial={},
    path/.initial={},
    color-left/.initial=FF00FF,
    color-right/.initial=0000FF,
    % types
    permanent/.code={\pgfkeyssetvalue{permanent}{#1}},
    sequence/.code={\pgfkeyssetvalue{sequence}{#1}},
    oneshot/.code={\pgfkeyssetvalue{oneshot}{#1}},
    innate/.code={\pgfkeyssetvalue{innate}{#1}},
    extra/.code={\pgfkeyssetvalue{extra}{#1}},
    % purchase stuff
    purchase/.code={\pgfkeyssetvalue{purchase}{#1}},
    % upgrade stuff
    upgrade/.code={\pgfkeyssetvalue{upgrade}{#1}},
    upgrade cost/.code={\pgfkeyssetvalue{upgrade cost}{#1}},
    % read-only key to get the printed type line
    type/.initial={%
        % supertypes
        \pgfkeysifdefined{innate}{Innate }{}%
        \pgfkeysifdefined{extra}{Forgotten }{}%
        % base types
        \pgfkeysifdefined{permanent}{Permanent }{}%
        \pgfkeysifdefined{sequence}{Sequence }{}%
    }
}

\newcommand{\setpath}[3]{
    \pgfkeys{/card/.cd,
        path=#1,
        color-left=#2,
        color-right=#3,
    }
}

\newsavebox{\rulestextbox}
\newlength{\savedfontsize}
\newlength{\savedheight}
\newlength{\extraheight}
\newcommand{\buildbox}[1][0mm]{}

\newlength{\textboxbottomoffset}
\setlength{\textboxbottomoffset}{0mm}

\newlength{\templen}

\makeatletter
\NewDocumentCommand{\card}{ +O{} }{

    % start a new page if we have to
    \clearpage

    % to make sure no arguments "leak" between different \card calls
    \begingroup
    % handle key-value arguments
    \pgfkeys{/card/.cd, #1}
    
    % render frame
    % the tikz stuff makes sure it's rendered in the background and not
    % as part of the document flow
    
    % background gradient:
    \definecolor{left}  {HTML}{\pgfkeysvalueof{/card/color-left}}
    \definecolor{right} {HTML}{\pgfkeysvalueof{/card/color-right}}
    \tikz[overlay, remember picture] 
    \node[
        anchor=north,
        yshift=-1.5mm,
        rectangle, 
        minimum width=\paperwidth-3mm, 
        minimum height=\paperheight-3mm,
        left color=left, 
        right color=right,
        shading = axis,
        shading angle=135, 
    ] at (current page.north) {};
    
    % frame (with transparency):
    \tikz[overlay, remember picture] 
    \node[
        anchor=north, 
        inner ysep=0pt,
        inner xsep=0pt,
        outer ysep=0pt,
        outer xsep=0pt,
    ] at (current page.north) {
        % \includegraphics[width=63mm]{frame2.png}
        \includegraphics[width=63mm]{%
            \pgfkeysifdefined{extra}{frame2-gray.png}{frame2.png}
        }
        
    };

    % cost
    \rendermana{\pgfkeysvalueof{/card/cost}}
    % \rendermanasymbol{1}{W}
    % \rendermanasymbol{2}{S}
    % \rendermanasymbol{3}{A}
    
    % make sure we have the right font ny default
    \bodyfont
    
    % name
    \bettertextbox
        [y=4.5mm, width=36mm, height=6mm, margin=0mm, fontsize=14pt, font=\titlefont\bfseries]
        [halign=center, valign=centerline]
    {%
        \pgfkeysvalueof{/card/name}
    }
     
    % type
    \bettertextbox
        [y=11mm, width=36mm, height=4mm, margin=0mm, fontsize=8pt, font=\titlefont\sbseries\itshape]
        [halign=center, valign=centerline]
    {%
        \pgfkeysvalueof{/card/type}
        % Basic -- \pgfkeysvalueof{/card/type}
    }

    % purchase cost?
    \newcommand{\purchaseunderlay}{%
    \begin{tcbclipinterior}
        \node[
            fit={(frame.north west) (frame.south east)},
            anchor=center,
            rectangle,
            rounded corners=1mm,
            inner sep=0pt,
            fill=black!25,
        ] at (frame.center) {};
    \end{tcbclipinterior}
    }
    \pgfkeysifdefined{purchase}{%
        \bettertextbox
            [%
                y=5mm, x=26.5mm,
                width=6mm, height=10.5mm, margin=1mm, 
                anchor=north east,
                fontsize=14pt, 
                font=\bfseries,
            ]
            [%
                halign=center, valign=center,
                underlay={\purchaseunderlay},
                underlay={
                    \begin{tcbclipframe}
                        \node[
                            anchor=north,
                            inner sep=0pt,
                            yshift=-1mm,
                            % opacity=0.5,
                        ] at (frame.north) {\includegraphics[height=3.5mm]{icons/padlock.png}};
                    \end{tcbclipframe}
                },
                top=5.5mm,
            ]
        {%
            \pgfkeysvalueof{purchase}%
        }
    }{%
        % not defined
    }

    % upgrade?
    \newcommand{\upgradeunderlay}{%
        \begin{tcbclipinterior}
            \node[
                fit={(frame.north west) (frame.south east)},
                anchor=center,
                rectangle,
                rounded corners=1mm,
                inner sep=0pt,
                fill=black!25,
            ] at (frame.center) {};
        \end{tcbclipinterior}
    }
    \pgfkeysifdefined{upgrade}{%
        \bettertextbox
            [%
                y=81.5mm, x=27mm,
                width=42mm, height=6mm, margin=1mm, 
                anchor=south east,
                fontsize=8pt,
            ]
            [%
                halign=center, valign=centerline, left=1mm, right=1mm,
                underlay={\upgradeunderlay},
            ]
        {%
            \pgfkeysvalueof{upgrade}%
        }
        \setlength{\templen}{%
            \directlua{%
                if string.len(\luastring{\pgfkeysvalueof{upgrade cost}}) > 1 then
                    tex.sprint("12pt")
                else
                    tex.sprint("14pt")
                end
            }
        }
        \bettertextbox
            [%
                y=81.5mm, x=-27mm,
                width=11mm, height=6mm, margin=1mm, 
                anchor=south west,
                fontsize=\templen,
                font=\bfseries,
            ]
            [%
                halign=center, valign=centerline,
                underlay={\upgradeunderlay},
                underlay={
                    \begin{tcbclipframe}
                        \node[
                            anchor=west,
                            inner sep=0pt,
                            xshift=1mm,
                            % opacity=0.5,
                        ] at (frame.west) {\includegraphics[height=4mm]{icons/upgrade.png}};
                    \end{tcbclipframe}
                },
                left=4.5mm,
            ]
        {%
            \pgfkeysvalueof{upgrade cost}%
        }
        \addtolength{\textboxbottomoffset}{6mm}
    }{%
        % not defined
    }

    % oneshot?
    \pgfkeysifdefined{oneshot}{%
        % defined 
        \tikz[overlay, remember picture] 
        \node[
            anchor=south, 
            inner ysep=0pt,
            inner xsep=0pt,
            outer ysep=0pt,
            outer xsep=0pt,
            % yshift=7.5mm,
            yshift=7mm+\textboxbottomoffset,
        ] at (current page.south) {
            % \includegraphics[width=36mm]{oneshot-banner.png}
            \includegraphics[width=30mm]{oneshot-banner.png}
        };
        \addtolength{\textboxbottomoffset}{5mm}
    }{%
        % not defined
    }

    % text
    \renewcommand{\buildbox}{%
        \sbox{\rulestextbox}{
            \setlength{\savedheight}{34mm - \textboxbottomoffset + \extraheight}
            \bettertextbox
                [%
                    y={81.5mm+\textboxbottomoffset}, 
                    width=54mm, 
                    % height=\pgfkeysifdefined{oneshot}{24mm}{32mm}, 
                    height=\savedheight, 
                    anchor=south, 
                    margin=2mm,
                ]
                [valign=center, bottom=0.25mm, % the bottom=... just adds some extra spacing there
                    % top=-2mm,
                    underlay=\pgfkeysifdefined{permanent}{%
                        \begin{tcbclipinterior}
                            \node[
                                anchor=center, 
                                opacity=0.15,
                                % yshift=1mm
                            ] at (frame.center) {
                                \includegraphics[height=14mm]{permanent-watermark.png}
                            };
                        \end{tcbclipinterior}
                    }{}
                ] 
            {%
                \pgfkeysifdefined{innate}{%
                    % {\centering\includegraphics[width=36mm]{innate-banner.png}\par}
                    \textit{Innate -- Play before the match starts.}\par%
                    % \textbf{Innate:} \textit{Play before the match starts.}\par%
                }{}%
                \pgfkeysvalueof{/card/text}%
                \global\setlength{\savedfontsize}{\tcbfitdim}
                \global\savedfontsize=\savedfontsize
            }%
        }
    }
    \setlength{\extraheight}{0mm}
    \buildbox
    \loop
    \ifdim\savedfontsize<7pt
        \setlength{\extraheight}{\extraheight + 2mm}
        \global\extraheight=\extraheight
        \buildbox
    \repeat
    % \ifdim\savedfontsize<7pt
    %     \setlength{\extraheight}{6mm}
    %     \buildbox
    % \else
    % \fi

    \usebox{\rulestextbox}
    % \the\savedfontsize

    % art:
    % (just a placeholder rectangle currently)
    \tikz[overlay, remember picture] 
    \node[
        anchor=north,
        yshift=-17mm,
        rectangle, 
        minimum width=36mm, 
        minimum height=30mm-\extraheight,
        left color=left, 
        right color=right,
        shading = axis,
        shading angle=135, 
        opacity = 0.3,
    ] at (current page.north) {};
    
    % path
    \bettertextbox
        [y=81.5mm, x=-27mm, width=24mm, height=4mm, anchor=north west, font=\titlefont\color{white}\sbseries, fontsize=7pt, margin=0.5mm]
        [valign=centerline, left=-0.5mm]
    {%
        \pgfkeysvalueof{/card/path}
    }
    
    % date - glorybound
    \bettertextbox
        [y=81.5mm, x=27mm, width=24mm, height=4mm, anchor=north east, font=\titlefont\color{white}\sbseries, fontsize=7pt, margin=0.5mm]
        [valign=centerline, right=-0.5mm, halign=flush right]
    {%
        \today\ \ --\ \ Glorybound
    }

    \endgroup
}
\makeatother



\begin{document}

\setpath{Berserker}{000000}{C8000F}

\card[%
    name={Bloodthirsty Axe},
    cost={SA},
    text={%
        \attack{2}

        I get +2 attack power if you are being attacked.
    },
]

\card[%
    name={Battle Rage},
    cost={},
    permanent,
    text={%
        At the start of each turn, if your opponent has scored at least three 
        points you may play an additional action this turn.
    },
    upgrade cost={6},
    upgrade={Gains: \textit{\textsb{Innate} \linebreak(Play before the match starts.)}}
]

\card[%
    name={Rite of Scars},
    cost={},
    text={%
        Gain \mana{XX}.

        Attack yourself \attackicon{} with an attack power of 2.
        
        \textit{(You are both attacking and being attacked. If you don't block, your opponent scores the points.)}
    },
    purchase={0},
    upgrade cost={6},
    upgrade={+1 attack power; +1 \mana{X} gained.},
]

\card[%
    name={Pillage},
    cost={},
    oneshot,
    text={%
        Banish a card your opponent played this turn.
    },
    purchase={2},
    upgrade cost={3},
    upgrade={... and another one of your opponent's cards of their choice.},
]

\card[%
    name={Retribution},
    cost={SAA},
    text={%
        \attack{2}

        I get +1 attack power for each two points your opponent has scored this match.

        I can't be blocked unless your \upgrade{opponent pays \mana{A}}{opponent pays \mana{AA}}.
    },
    purchase={4},
    upgrade cost={4},
    % upgrade={Your opponent must pay \mana{AA} instead.},
]

\setpath{Fireheart}{F44B02}{F9028A}
\newcommand{\burn}[1]{\textcolor[HTML]{F22643}{\textbf{#1}}}

\card[%
    name={Insatiable Flame},
    cost={W},
    text={%
        \attack{1}

        I \burn{burn} your opponent when I hit. \textit{(This can help you set them ablaze)}
    },
]

\card[%
    name={Set Ablaze},
    cost={WWA},
    oneshot,
    text={%
        I cost no resources once your opponent has been \burn{burned} twice.
    
        \attack{4}
    },
]

\card[%
    name={Inferno Circle},
    cost={},
    sequence,
    text={%
        \sequence
        [If your opponent attacked this turn, I \burn{burn} them.]
        [If your opponent attacked this turn, I \burn{burn} them.]
    },
]

\card[%
    name={Kindled Flame},
    cost={},
    innate,
    permanent,
    text={%
        When your opponent is \burn{burned} for the third time play Set Ablaze immediately as a special action. If it is banished or forgotten, recall it first.
    },
]

\card[%
    name={Pyre Offering},
    cost={},
    sequence,
    oneshot,
    text={%
        \sequence
        [Banish a non-Fireheart card from your hand to recall Consumed by Flame.]
        [You may play an additional action.]
    },
]

\card[%
    name={Consumed by Flame},
    cost={},
    extra,
    text={%
        \textit{This card begins the match forgotten and is recalled by Pyre Offering}

        Consumed by Flame is a copy of the card that was banished by Pyre Offering, except that it \burn{burns} your opponent when it hits and when it blocks their attacks.
    },
]

\setpath{Legionnaire}{FFA500}{AF0301}

\card[%
    name={Balanced Blade},
    cost={AA},
    text={%
        \attack{2}

        Recall Parry.
    },
    upgrade cost={4},
    upgrade={Costs \mana{A} instead of \mana{AA}.},
]

\card[%
    name={Parry},
    cost={A},
    linked={Balanced Blade},
    oneshot,
    text={%
        I cost one resource less if you played Balanced Blade last turn.
        
        \block
    },
]

\card[%
    name={Arrow Volley},
    cost={SF},
    text={%
        Repeat the following twice:
        \attack{1}
        
        \textit{(One block can only stop one attack.)}
    },
    upgrade cost={6},
    upgrade={Attacks three times instead of two.},
]

\card[%
    name={Decisive Strike},
    cost={SAA},
    text={%
        I cost one resource less for each two points you've scored this match.
        
        \attack{3}
    },
    purchase={0},
    upgrade cost={4},
    upgrade={+1 attack power.},
]

\card[%
    name={Battle Tactics},
    cost={F},
    innate,
    permanent,
    text={%
        Choose a strategy:

        \textit{Skirmish} -- You may play an additional action on the \upgrade{second}{second and fourth} turn.

        \textit{Ambush} -- The first time you attack, give an attack \upgrade{+2}{+4} attack power.

        \textit{Besiege} -- Your opponent pays \upgrade{\mana{A}}{\mana{AA}}.
    },
    purchase={3},
    upgrade cost={4},
    % upgrade={\{second and fourth\}; \{+4\}; \{\mana{AA}\}},
]

\card[%
    name={Shield Wall},
    cost={S},
    permanent,
    text={%
        First turn:
        \block
        
        Each turn, if you are attacking, block an attack with less attack power than yours.
    },
    purchase={5},
]

\setpath{Dancer}{5EF285}{F48DD2}

\card[%
    name={Elegant Reversal},
    cost={SF},
    text={%
        \block

        When I block an attack, I attack back with equal attack power.
    },
]

\card[%
    name={Mesmerize},
    cost={},
    sequence,
    text={%
        \sequence
            [No effect]
            [Your opponent's actions cost twice as many resources. 
            
            \textit{(For example, an action that would cost \mana{FAA} costs \mana{FFAA}.)}]
    },
]

\card[%
    name={Dizzying Spin},
    cost={F},
    sequence,
    text={%
        \sequence
            [\attack{1}\par If I hit, name a card.]
            [Your opponent can't play the named card(s).]
    },
]

\card[%
    name={Just Out of Reach},
    cost={A},
    text={%
        Your opponent's attacks get -2 attack power this turn.
    },
]

\card[%
    name={Entice to Dance},
    cost={A},
    innate,
    permanent,
    text={%
        Your opponent recalls Try to Keep Up.
        
        When the match ends, score three points unless your opponent has played Try to Keep Up.
    },
]

\card[%
    name={Try to Keep Up},
    cost={},
    extra,
    text={%
        \textit{This card begins the match forgotten and is given to your opponent by Entice to Dance.}

        Banish Try to Keep Up on top of Entice to Dance.
    },
]

\setpath{Arcanist}{E100FF}{5300EF}

\card[%
    name={Arcane Research},
    cost={F},
    text={%
        Choose two random advanced cards you have not yet learned and recall them. If you recall an innate card, play it immediately.
    },
]

\card[%
    name={Forcefield},
    cost={W},
    text={%
        \block

        For the rest of the match, Forcefield costs no resources.
    },
]

\card[%
    name={Meteor Invocation},
    cost={WA},
    sequence,
    text={%
        \sequence
            [No effect.]
            [\attack{4}]
    },
]

\card[%
    name={Shrinking Ray},
    cost={WA},
    oneshot,
    text={%
        Your opponent pays \mana{AA} if able, or \mana{A} if that's all they have. They must undo an action they played this turn if doing so would make them more able to pay. \textit{(Undone actions are recalled and refunded and have no effect.)}
    },
]

\card[%
    name={Spellweaving},
    cost={},
    innate,
    permanent,
    text={%
        Reveal an Arcanist card and a non-Arcanist card from your hand. When you play one of those cards, play the other as a special action if able.
    },
]

\card[%
    name={All-Consuming Void},
    cost={A},
    extra,
    text={%
        \textit{Forbidden Magic - This card begins the match forgotten and can be recalled by an upgraded Arcane Research.}

        At end of turn, end the match. \textit{(The player with more points wins)}
    },
]

\card[%
    name={True Reincarnation},
    cost={},
    extra,
    text={%
        \textit{Forbidden Magic - This card begins the match forgotten and can be recalled by an upgraded Arcane Research.}

        If your opponent would score points this turn, instead they score no points and your resources are restored to your starting resources.
    },
]

\card[%
    name={Awaken the Old Gods},
    cost={AAA},
    extra,
    sequence,
    text={%
        \textit{Forbidden Magic - This card begins the match forgotten and can be recalled by an upgraded Arcane Research.}

        \sequence
            [\attack{4}]
            [\attack{4}]
    },
]

\setpath{Assassin}{000000}{FF00AF}

\card[%
    name={Backstab},
    cost={F},
    text={%
        If your opponent isn't attacking: 

        \attack{3}
    },
]

\card[%
    name={Shadowstep},
    cost={},
    text={%
        \block
        
        Banish me unless you pay \mana{F}.
    },
]

\card[%
    name={Trap Setting},
    cost={F},
    text={%
        Secretly choose a Trap and put it into play face down.
        You can turn that card face up to play it as a special action any time after this turn ends.
    },
]

\newcommand{\traptext}{\textit{Trap - This card begins the match forgotten and is put into play by Trap Setting.}}

\card[%
    name={Poisoned Dagger},
    cost={},
    extra,
    text={%
        \traptext

        If you hit with an attack this turn, I become permanent and at the start of each turn, you score one point. Otherwise, banish me.
    },
]

\card[%
    name={Smoke Shroud},
    cost={},
    extra,
    oneshot,
    text={%
        \traptext
        
        If your opponent didn't attack last turn, they can't score points this turn.
    },
]

\card[%
    name={Feign Defeat},
    cost={},
    extra,
    oneshot,
    text={%
        \traptext

        If your opponent has more points than you, they can't block this turn.
    },
]

\card[%
    name={Always More Knives},
    cost={FAA},
    sequence,
    text={%
        \sequence
            [\attack{3}]
            [\attack{1}]
    },
]

\card[%
    name={Hidden in Plain Sight},
    cost={},
    innate,
    permanent,
    text={%
        After the first turn of the match, but before the second, there is a secret turn. Your opponent and their cards do not participate in this turn, and no points can be scored.
    },
]

\setpath{Windwalker}{044CF4}{3EF60D}

\card[%
    name={Tempest Blade},
    cost={AA},
    text={%
        \attack{2}

        If Storm-Infused Blade is blocked, you may pay \mana{WA} to score two points.
    },
]

\card[%
    name={Leap Skyward},
    cost={F},
    sequence,
    text={%
        \sequence
            [\block]
            [Your attacks get +1 attack power.]
    },
]

\card[%
    name={Storm Surge},
    cost={},
    text={%
        Gain \mana{X} until you have more resources than your opponent.
    },
]

\card[%
    name={Wind's Favor},
    cost={A},
    innate,
    permanent,
    text={%
        As long as both players are attacking, your attacks get +1 attack power.
    },
]

\card[%
    name={Bring the Lightning},
    cost={},
    innate,
    permanent,
    text={%
        Your first attack of the match  gets +1 attack power. Then note its attack power and recall Echoing Thunder.
    },
]

\card[%
    name={Echoing Thunder},
    cost={},
    extra,
    oneshot,
    text={%
        \textit{I begin the match forgotten and am recalled by Bring the Lightning.}

        My base attack power X is the number you noted for Bring the Lightning.

        \attack{X}
    },
]

\setpath{Hammer Priest}{33CCFF}{FECD01}

\card[%
    name={Smite},
    cost={SW},
    text={%
        \attack{3}
    },
    upgrade cost={6},
    upgrade={Banish any non-innate card from play.},
]

\card[%
    name={Shield of Faith},
    cost={S},
    sequence,
    text={%
        \sequence
            [\block]
            [All points you score are doubled.]
    },
    upgrade cost={3},
    upgrade={When Shield of Fate blocks, it block all attacks instead of just one.},
]

\card[%
    name={Divine Intervention},
    cost={},
    oneshot,
    text={%
        You can't play me until you've scored at least four points.

        \block

        Score one point.
        
        Gain \mana{SW}.
    },
    purchase={0},
    upgrade cost={5},
    upgrade={For the rest of the match, your attacks get +2 attack power.},
]

\card[%
    name={Desperate Prayer},
    cost={},
    oneshot,
    text={%
        When you forget me, return me to your hand. \textit{(You will begin the match with six cards in hand.)}

        Score one point.
    },
    purchase={3},
]

\card[%
    name={Righteous Conviction},
    cost={},
    innate,
    permanent,
    text={%
        Gain \mana{X}.

        While your opponent has more points than you, they can't block an attack unless they pay \mana{A}.
    },
    purchase={6},
]

\setpath{Druid}{AA0CFF}{00D025}
% \setpath{Druid}{00DB1D}{BF00FF}
% \setpath{Druid}{BF00FF}{00DB1D}

\card[%
    name={Bind in Thorns},
    cost={WA},
    text={%
        \attack{1}

        \block
    },
    upgrade cost={4},
    upgrade={+1 attack power.},
]

\card[%
    name={Lifebloom},
    cost={},
    sequence,
    text={%
        \sequence
            [Gain \mana{X}.]
            [Gain \mana{X}.]
    },
    upgrade cost={6},
    upgrade={Gain \mana{XX} each turn instead.},
]

\card[%
    name={Wildshape},
    cost={},
    text={%
        Choose, reveal and recall an Animal Form.
    },
    purchase={0},
    upgrade cost={4},
    upgrade={Gains: \textit{\textsb{Innate} \linebreak(Play before the match starts.)}}
]

\card[%
    name={Tooth and Claw},
    cost={S},
    linked={Wildshape},
    linked type={Animal Form},
    oneshot,
    text={%
        \attack{3}

        I can't be blocked unless your opponent pays \mana{A}.
    },
]

\card[%
    name={Take Wing},
    cost={S},
    linked={Wildshape},
    linked type={Animal Form},
    sequence,
    oneshot,
    text={%
        \sequence
            [\block]
            [\block]
    },
]

\card[%
    name={Fierce Seedling},
    cost={A},
    permanent,
    text={%
        Each turn, I grow. Then, if I have grown at least three times:

        \attack{1}
    },
    purchase={3},
]

\card[%
    name={Sheltering Ancient},
    cost={},
    innate,
    permanent,
    text={%
        When you are attacked, banish me to block that attack.
        At the end of the second turn of the match, banish me.
    },
    purchase={5},
]

% \setpath{XXX}{aa00aa}{0000aa}
% \card[%
%     name={},
%     cost={},
%     text={%
%         
%     },
% ]


\setpath{TEST CARD}{aa00aa}{aaaa00}

\card[%
    name={FAKE TEST CARD},
    cost={AAA},
    linked={an upgraded Arcane Research},
    oneshot,
    text={%
        \attack{4}

        \attack{4}
    },
]

\card[%
    name={FAKE TEST CARD},
    cost={AAA},
    linked={an upgraded Arcane Research},
    oneshot,
    text={%        
        \sequence
            [\block]
            [\attack{4}]
    },
]


\end{document} 
