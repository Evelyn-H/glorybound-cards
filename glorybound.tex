% !TeX program = lualatex

\documentclass{article}
% \documentclass{minimal}
\usepackage[
    paperwidth=63mm, 
    paperheight=88mm,
    margin=0mm,
]{geometry}
\setlength\parindent{0pt}
\setlength{\parskip}{2mm}

\usepackage{xcolor}

% makes lua stuff easier
\usepackage{luacode}

% for \NewDocumentCommand
\usepackage{xparse}

% for rendering the SVG frame
% \usepackage{svg}

% for filler text
\usepackage{lipsum}

% for getting the current date
\usepackage[datesep=/]{datetime2}
\DTMsetdatestyle{mmddyy}

% for textboxes
% this also pulls in tikz as a dependency
\usepackage[fitting, skins]{tcolorbox}

% for length maths shenanigans
\usepackage{calc}

% Some tcolorbox hackery to add "proper" one-line vertical centering
\newlength{\Eheight}
\makeatletter
\def\tcb@dbox@top#1#2#3#4#5{\pgftext[x=#1,y=#2+#3,left,top]{#5}}
% define new valign type
\def\tcb@dbox@centerline#1#2#3#4#5{%
    % based on valign=top 
    \pgftext[x=#1,y=#2+#3,left,top]{%
        % make sure we have the correct front set for measuring 
        \pgfkeysvalueof{/textbox/font}%
        \fontsize{\tcbfitdim}{\tcbfitdim}\selectfont%
        % measure cap height
        \setlength{\Eheight}{\heightof{E}}%
        % \the\Eheight
        % and then raise (actually lower) the text based on that
        \raisebox{%
            0pt-0.5\Eheight-(\pgfkeysvalueof{/textbox/height}-\pgfkeysvalueof{/textbox/margin}-\pgfkeysvalueof{/textbox/margin})/2
        }{#5}
    }
}
\def\tcb@dbox@center#1#2#3#4#5{\pgftext[x=#1,y=#2+#3/2,left]{#5}}
\def\tcb@dbox@bottom#1#2#3#4#5{\pgftext[x=#1,y=#2,left,bottom]{#5}}

\tcbset{
    % add new valign type to the list of valid options
    valign/centerline/.code={\def\kvtcb@valignupper{centerline}}
}
\makeatother

% for fonts
\usepackage{fontspec}
\newfontfamily\bodyfont{CrimsonPro-Regular}[
    Path           = "fonts/Crimson_Pro/",
    Extension      = .ttf,
    Ligatures      = TeX,
    BoldFont       = CrimsonPro-Bold,
    ItalicFont     = CrimsonPro-Italic,
    BoldItalicFont = CrimsonPro-BoldItalic,
]
\newfontfamily\titlefont{Grenze-Regular}[
    Path           = "fonts/Grenze/",
    Extension      = .ttf,
    Ligatures      = TeX,
    BoldFont       = Grenze-Bold,
    ItalicFont     = Grenze-Italic,
    BoldItalicFont = Grenze-BoldItalic,
    FontFace       = {sb}{n}{Grenze-SemiBold},
    FontFace       = {sb}{it}{Grenze-SemiBoldItalic},
]

\DeclareOldFontCommand{\sbseries}{\fontseries{sb}\selectfont}{\mathbf}
\DeclareTextFontCommand{\textsb}{\sbseries}


\begin{document}

\bodyfont

% build different frames
\directlua{frames = require("frames")}
\directlua{frames.build()}

% render frame
% the tikz stuff makes sure it's rendered in the background and not
% as part of the document flow

% background gradient:
\definecolor{left} {HTML}{FFA500}
\definecolor{right} {HTML}{AF0301}
\tikz[overlay, remember picture] 
\node[
    anchor=north,
    yshift=-1.5mm,
    rectangle, 
    minimum width=\paperwidth-3mm, 
    minimum height=\paperheight-3mm,
    left color=left, 
    right color=right,
    shading = axis,
    shading angle=135, 
] at (current page.north) {};

% frame (with transparency):
\tikz[overlay, remember picture] 
\node[
    anchor=north, 
    inner ysep=0pt,
    inner xsep=0pt,
    outer ysep=0pt,
    outer xsep=0pt,
] at (current page.north) {
    % \includesvg[width = 63mm]{frames/frame-berserker.svg}
    \includegraphics[width=63mm]{frame2.png}
};


\pgfkeys{/textbox/.cd,% set the initial path
    anchor/.initial=north,
    anchorpos/.initial=north,
    x/.initial=0mm,
    y/.initial=0mm,
    width/.initial=54mm,
    height/.initial=4mm,
    margin/.initial=1mm,
    fontsize/.initial=10pt,
    font/.initial=\bodyfont,
}

\makeatletter
% first argument (optional): key-value options
% second argument (optional): extra stuff passed to tcolorbox
%third argument: textbox contents
\NewDocumentCommand{\bettertextbox}{ O{} O{} +m }{
    \begingroup
    % handle key-value arguments
    \pgfkeys{/textbox/.cd, #1}

    \tikz[overlay, remember picture] 
    \node[
        anchor=\pgfkeysvalueof{/textbox/anchor}, 
        xshift=\pgfkeysvalueof{/textbox/x},
        yshift= -1 * \pgfkeysvalueof{/textbox/y},
        inner ysep=0pt,
        inner xsep=0pt,
        outer ysep=0pt,
        outer xsep=0pt,
    ] at (current page.\pgfkeysvalueof{/textbox/anchorpos}) {%
        \begin{tcolorbox}[
            % show bounding box=green,
            width=\pgfkeysvalueof{/textbox/width},
            height=\pgfkeysvalueof{/textbox/height},
            fit,
            fit basedim=\pgfkeysvalueof{/textbox/fontsize},
            fit skip=1.2, % baselineskip
            blank,
            parbox=false, % makes parskip work
            halign=flush left,
            boxsep=\pgfkeysvalueof{/textbox/margin},
            % fontupper=\fontsize{\pgfkeysvalueof{/textbox/fontsize}}{\pgfkeysvalueof{/textbox/fontsize}}\selectfont,
            % tikz={opacity=0.5,transparency group},
            #2
        ]%%
            \pgfkeysvalueof{/textbox/font}%
            #3
        \end{tcolorbox}%
    };
    \endgroup
}
\makeatother


% \newlength{\boxmargin}\setlength{\boxmargin}{0mm}

\bettertextbox
    [y=4.5mm, width=36mm, height=6mm, margin=0mm, fontsize=14pt, font=\titlefont\bfseries]
    [halign=center, valign=centerline]
{%
    Battle Rage
    % Shelter at the Crossroads
}
 
\bettertextbox
    [y=11mm, width=36mm, height=4mm, margin=0mm, fontsize=9pt, font=\titlefont\sbseries\itshape]
    [halign=center, valign=centerline]
{%
    Innate Permanent
}

\bettertextbox
    [y=81.5mm, width=54mm, height=32mm, anchor=south, margin=2mm]
    [valign=center, bottom=0.25mm, % the bottom=... just adds some extra spacing there
        underlay={
            \begin{tcbclipinterior}
                \node[anchor=center, opacity=0.1] at (frame.center) {
                    \includegraphics[height=14mm]{permanent-watermark.png}
                };
            \end{tcbclipinterior}
        }
    ] 
{%
    % \textbf{Permanent:} Stays in play.
    % 
    At the start of each turn, if your opponent has scored at least two points you may play an additional action this turn.
}

\bettertextbox
    [y=81.5mm, x=-27mm, width=24mm, height=4mm, anchor=north west, font=\titlefont\color{white}\sbseries, fontsize=7pt, margin=0.5mm]
    [valign=centerline, left=-0.5mm]
{%
    Berserker
}

\bettertextbox
    [y=81.5mm, x=27mm, width=24mm, height=4mm, anchor=north east, font=\titlefont\color{white}\sbseries, fontsize=7pt, margin=0.5mm]
    [valign=centerline, right=-0.5mm, halign=flush right]
{%
    \today\ \ --\ \ Glorybound
}

% using luacode* because there are percentage signs
% declare the sqrt function in Lua
% \begin{luacode*}
%     function compute_sqrt(v)
%         local s = math.sqrt(tonumber(v))
%         local s_f = math.floor(s)
%         local o
%         if (math.abs(s_f * s_f - v) < 1.0e-5) then
%             o = string.format("%.1f", s)
%         else
%             o = string.format("%.6f", s)
%         end
%         tex.print(o)
%     end
% \end{luacode*}

% % declare a wrapper in TeX
% \newcommand{\luasqrt}[1]{\directlua{compute_sqrt(#1)}}

% \luasqrt{5}


\end{document} 
