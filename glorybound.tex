% !TeX program = lualatex

\documentclass{article}
% \documentclass{minimal}
\usepackage[
    paperwidth=63mm, 
    paperheight=88mm,
    margin=0mm,
]{geometry}
\setlength\parindent{0pt}
\setlength{\parskip}{2mm}

% \usepackage{xcolor}

% makes lua stuff easier
\usepackage{luacode}

% for \NewDocumentCommand
\usepackage{xparse}

% for rendering the SVG frame
\usepackage{svg}

% for filler text
\usepackage{lipsum}

% for textboxes
\usepackage[fitting, skins]{tcolorbox}

\usepackage{calc}

% for fonts
\usepackage{fontspec}
\newfontfamily\bodyfont{CrimsonPro-Regular}[
    Path           = "fonts/Crimson_Pro/",
    Extension      = .ttf,
    Ligatures      = TeX,
    BoldFont       = CrimsonPro-Bold,
    ItalicFont     = CrimsonPro-Italic,
    BoldItalicFont = CrimsonPro-BoldItalic,
]
\newfontfamily\titlefont{Grenze-Regular}[
    Path           = "fonts/Grenze/",
    Extension      = .ttf,
    Ligatures      = TeX,
    BoldFont       = Grenze-Bold,
    ItalicFont     = Grenze-Italic,
    BoldItalicFont = Grenze-BoldItalic,
    FontFace       = {sb}{n}{Grenze-SemiBold},
]

\DeclareOldFontCommand{\sbseries}{\fontseries{sb}\selectfont}{\mathbf}
\DeclareTextFontCommand{\textsb}{\sbseries}


\begin{document}

\bodyfont

% build different frames
\directlua{frames = require("frames")}
\directlua{frames.build()}

% render frame
% the tikz stuff makes sure it's rendered in the background and not
% as part of the document flow
\tikz[overlay, remember picture] 
\node[
    anchor=north, 
    inner ysep=0pt
] at (current page.north) {
    \includesvg[width = 63mm]{frames/frame-berserker.svg}
};


\newlength{\boxmargin}\setlength{\boxmargin}{1mm}

\NewDocumentCommand{\textbox}{ O{} O{} +m }{
% \newcommand{\textbox}[2][]{
    \tikz[overlay, remember picture] 
    \node[
        anchor=north, 
        yshift=0cm,
        xshift=0cm,
        inner ysep=0pt,
        #1
    ] at (current page.north) {
        \begin{tcolorbox}[
            % show bounding box=blue,
            fit,
            fit basedim=10pt,
            blank,
            parbox=false, %makes parskip work
            halign=flush left,
            boxsep=\boxmargin,
            % fontupper=\fontsize{10pt}{10pt}\selectfont,
            #2
        ]%%
        #3
        \end{tcolorbox}
    };
}


\NewDocumentCommand{\textboxcenter}{ m O{} O{} +m }{
    \textbox[#2][#3, height=#1]{%%
        \raisebox{(\heightof{E})/(-2) - (#1-2\boxmargin)/2}[0pt]{%%
            #4
        }
    }
}


% \textbox[yshift=-4.5mm][%%
%     width=54mm, height=7mm, halign=center, valign=center,
% % ]{\textbf{Battle Rage}}
% ]{Battle Rage}

\textboxcenter{7mm}[yshift=-4.5mm][%%
    width=54mm, halign=center, %valign=center,
    fit basedim=14pt,
]{\titlefont\textbf{Battle Rage}}
% ]{Battle Rage}

% \textbox[yshift=-12mm][%%
%     width=54mm, height=5mm, %valign=center,
% ]{%%
%     % \raisebox{(\heightof{E}-.5\height)/2}{Basigc}
%     % \raisebox{.5\height-(\heightof{x})/2}{Basic Permanent}
%     \raisebox{(\heightof{E})/(-2)-2.5mm}[0pt]{%
%         Basic Permanent
%     }
% }

\textboxcenter{5mm}[yshift=-12mm][%%
    width=54mm,
]{%%
    \titlefont \sbseries Basic Permanent
    % \titlefont Basic Permanent
}

\textbox[yshift=-17.5mm][%%
    width=54mm, height=64mm, valign=center,
]{
    \textbf{Permanent:} Stays in play.

    At the start of each turn, if your opponent has scored at least two points you may play an additional action this turn.
}

% using luacode* because there are percentage signs
% declare the sqrt function in Lua
% \begin{luacode*}
%     function compute_sqrt(v)
%         local s = math.sqrt(tonumber(v))
%         local s_f = math.floor(s)
%         local o
%         if (math.abs(s_f * s_f - v) < 1.0e-5) then
%             o = string.format("%.1f", s)
%         else
%             o = string.format("%.6f", s)
%         end
%         tex.print(o)
%     end
% \end{luacode*}

% % declare a wrapper in TeX
% \newcommand{\luasqrt}[1]{\directlua{compute_sqrt(#1)}}

% \luasqrt{5}


\end{document} 
